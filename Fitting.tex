\subsection{Fit Procedure}
\label{sec:FitProcedure}

In this section, we discuss ... 

A crucial part of the analysis is the treatment of residual correlations in the fit procedure.

\subsubsection{Fitting via the transform method}

%Proper fitting techniques for this analysis are still a work in progress.  
%In particular, efforts are being made to evaluate the extent to which residual correlations permeate the measured correlation functions.  
In this analysis, the importance of residual correlations can be characterized by two sets of parameters.  
The scattering lengths $f_0$ of the residual pair describe the strength of the residual pair interaction, such that a larger $\abs{f_0}$ will generally lead to a more pronounced residual correlation function.  
Meanwhile, the pair purity $\lambda$ of each pair type describes the relative contribution of each individual correlation function (i.e. primary or residual) to the total correlation function, as described in Section \ref{sec:Residual}. 
The following sections will discuss how these two parameters are evaluated in this analysis.


%Discussion of current status of fit results and systematic uncertainties.  
%Report measured systematics as well as possible sources of systematics yet to be explored.



\subsubsection{Scattering parameters of residual pairs}
\label{sec:ScatteringParams}

As previously mentioned, little information exists about the hyperon-hyperon scattering parameters.  
One goal of this analysis is to make a measurement of the $\Lambda\Lambda$ scattering lengths and effective radius.  
However, the task remains to decide how to deal with the scattering parameters of all the various residual correlations, which are equally unknown.  
One option would be to include separate parameters for each pair interaction.  
However, this would introduce two or three new parameters (for particle-particle or particle-antiparticle, respectively) per pair type and yield a very unconstrained fit.  
Another option would be to tighten cuts to remove as many of the secondary $\Lambda$ as possible.  However this route stands to significantly reduce the statistics of the fit, as primary $\Lambda$ will be lost alongside the secondary.  
On top of that, reconstruction cuts are unlikely to appreciably cut out $\Sigma$ daughters, as the electromagnetic decay of the $\Sigma$ comes with a very short decay length.

Yet another option would be to ansatz that all the pairs interact with approximately the same strength as the $\Lambda\Lambda$ interaction.  
The disadvantage of this method is that the final measurement will essentially yield a weighted average of the various pair scattering parameters.  
However, the advantage is that it allows all pairs (primary-primary, primary-secondary, secondary-secondary) to provide relevant contributions to the correlation function.  
One potential way to estimate the systematic uncertainties in this method would be to do several additional fits, wherein the scattering parameters of the other pairs are treated as being 50\%, 25\%, and 0\% (no interaction) the strength of the $\Lambda\Lambda$ parameters.

\subsubsection{Minuit fitting}
\label{sec:MinuitFit}

%Include practical treatment of fit parameters here. Residual correlation scattering lengths, simultaneous fitting.  Normalization factor vs polynomial background.
Discussion of Minuit fitting procedure will go here.

