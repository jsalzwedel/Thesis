\subsection{Fit procedure}
\label{sec:FitProcedure}

In this section, we discuss the procedure by which we fit the data.
We use a self-developed software package called FemtoFitter.
At its heart, FemtoFitter utilizes the ROOT TMinuit implementation of the Minuit fitting package.

% Description of Minuit, with reference to the paper.
% Fit function

% chisquare vs log-likelihood (preliminary trials showed that they gave similar results. The chisquare version ran much faster, so we stuck with that.


% applying smear matrices


% normalization parameters
\subsubsection{Fit parameters}
% fixing parameters
% fit range
% fitting background range
% 
% joint fitting

% \subsubsection{Scattering parameters of residual pairs}

\subsubsection{Scattering parameters of residual pairs}
\label{sec:ScatteringParams}

% Revamp this

As previously mentioned, little information exists about the hyperon-hyperon scattering parameters.  
One goal of this analysis is to make a measurement of the $\Lambda\Lambda$ scattering lengths and effective radius.  
However, the task remains to decide how to deal with the scattering parameters of all the various residual correlations, which are equally unknown.  
One option would be to include separate parameters for each pair interaction.  
However, this would introduce two or three new parameters (for particle-particle or particle-antiparticle, respectively) per pair type and yield a very unconstrained fit.  
Another option would be to tighten cuts to remove as many of the secondary $\Lambda$ as possible.  However this route stands to significantly reduce the statistics of the fit, as primary $\Lambda$ will be lost alongside the secondary.  
On top of that, reconstruction cuts are unlikely to appreciably cut out $\Sigma$ daughters, as the electromagnetic decay of the $\Sigma$ comes with a very short decay length.

Yet another option would be to ansatz that all the pairs interact with approximately the same strength as the $\Lambda\Lambda$ interaction.  
The disadvantage of this method is that the final measurement will essentially yield a weighted average of the various pair scattering parameters.  
However, the advantage is that it allows all pairs (primary-primary, primary-secondary, secondary-secondary) to provide relevant contributions to the correlation function.  
One potential way to estimate the systematic uncertainties in this method would be to do several additional fits, wherein the scattering parameters of the other pairs are treated as being 50\%, 25\%, and 0\% (no interaction) the strength of the $\Lambda\Lambda$ parameters.

\subsubsection{Fit range and background}
\label{sec:FitBackground}

% Normalization factor and polynomial background.


