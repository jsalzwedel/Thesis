\section{Introduction}

% Thesis statement. 

Particle colliders allow us to study quantum chromodynamics (QCD), 


One major focus of particle accelerator physics has been heavy-ion collisions.
At LHC energies, where the ions have center of mass energies per nucleon on the order of TeV, these collisions can produce a hot, dense medium from which thousands of particles are emitted \cite{...}.
This medium is colloquially called a fireball, with a more general name being "particle-emitting source" or even just "event". 

For a short time after the initial collision (\~fm/$c$), there exists a hot, dense medium called the quark-gluon plasma \cite{...}.
Over time the plasma expands and cools, which leads to hadronization.
In the subsequent hadronic phase, the hadrons scatter off each other in an elaborate billiard ball game, with particles near the edge of the expanding fireball having the possibility to free-stream to the detector.
This hadronic rescattering phase lasts for tens of fm/$c$ \cite{...}.

Eventually, the source becomes dilute enough that the collisions die off and the particles free-stream to the detector.
While people often speak of a singular kinetic freezeout time (the time when the rescatterings cease), in actuality particles are emitted at varying rates throughout the lifetime of the fireball.


The primary goal of heavy-ion physics is to study the evolution of these fireballs.
The field seeks to answer questions such as "When does hadronization occur?", "What equations of state describe the evolution of source?", "What interactions are occuring between particles during during this evolution?", and "What is the size and shape of the source at kinetic freezeout?".
A number of models have been developed to probe these questions and others \cite{...,...,...,...}.

Very loosely speaking, a model of a collision has three main components: initial conditions describing the shape and energy density of the source; a theory describing how the system evolves over time; and the final result of collision, namely a description of the particles that are emitted, including their momenta, emission time, etc.
Collider experiments have been designed so as to simplify assumptions about the initial conditions of fireballs.
Experiments have used ions such as $\ce{^197 \mathrm{Au}^79}$ and $\ce{^208 \mathrm{Pb}^82}$, since these particles have a spherical symmetry to their nuclear structure that yields relatively simple initial conditions.
Particle collider experiments provide insight into the final state of collisions by measuring the final momenta of particles produced in millions of events.



%In the last 15 years, the Relativistic Heavy Ion Collider (RHIC) has performed Au--Au and U--U collisions, and the Large Hadron Collider (LHC) has done Pb--Pb collisions.




%By looking at the final-state information of large ensembles of collisions and making various model assumptions about the initial conditions and evolution, it is possible to work backward to obtain a better grasp of what happens in the early stages of the collisions.
%To answer many of the above questions, it is often necessary to look at this final-state information and work backwards.

% 


Beyond simply measuring asymptotic particle momenta, collider experiments can use various analysis techniques to probe other facets of the fireball evolution.
These techniques include flow analyses \cite{...}, ..., and an interferometry technique known as femtoscopy \cite{...}.


% Pb--Pb - Doubly magic

% Interest in understanding the dynamics of the matter created in heavy-ion physics. How does size/shape evolve? How do the particles interact?

% In this thesis...
In this note, we discuss an interferometry technique known as femtoscopy, which is capable of shedding light on both of these questions.




\subsection{Femtoscopy (HBT)}

Hanbury-Brown and Twiss first used two-particle interferometry of photons to measure the angular size of the star Sirius \cite{HanburyBrown:1956bqd}.
Later, it was observed that similar techniques could be employed in a laboratory setting with pions to characterize the size of particle collisions \cite{Goldhaber:1960sf}.
This technique of measuring two-particle correlations for particle pairs with small relative momentum is often referred to as HBT.
HBT was originally used to study photons and later used to look at pions, so it often carries a connotation of Bose-Einstein correlation effects.
Modern analyses look at all manner of hadrons, and therefore the correlations might be subject to bosonic, fermionic, or non-identical particle statistics.
As the focus of these analyses in the particle collision context is on measuring very small sources, the more general term \textit{femtoscopy} has come into use.

% chaotic source
\subsection{Femtoscopic sources}

Femtoscopic analyses are capable of measuring spatial and temporal characteristics of heavy-ion collisions.



 with a particular emphasis made on estimating the homogeneity lengths (also called radii) of particle-emitting sources.  
These analyses traditionally study pions \cite{Goldhaber:1960sf,Aamodt:2011mr} because of their ready availability and sensitivity to the spatial scale of the source.  
However, measurements of heavier particles such as kaons \cite{Abelev:2012ms} and baryons \cite{Gos:2007cj} can serve to complement the pion results.  
One motivation for studying an assortment of heavier particles is to test the hydrodynamic prediction that radial flow should cause the source radii to scale with the transverse mass $m_{\mathrm{T}}$ of the particles \cite{Csorgo:1995bi,Lisa:2005dd}.

\subsection{Hadronic interactions}
Baryon--(anti)baryon correlation functions are also useful in the study of final state interactions (FSI).  
Measurements of scattering lengths of pairs such as p$\Lambda$ and $\Lambda\Lambda$ are of interest for rescattering calculations, and, in the latter case, for understanding the properties of neutron stars \cite{SchaffnerBielich:2008kb,Wang:2010gr}.  
Baryon-antibaryon correlations allow for the study of annihilation processes.  

\subsection{$\Lambda$ femtoscopy}
In this note, we present two-$\Lambda$ momentum correlation functions measured in PbPb collisions at $\sqrt{s_{NN}}=2.76$ TeV by the ALICE Collaboration.  
Results are shown for $\Lambda\Lambda$, $\bar{\Lambda}\bar{\Lambda}$ and $\Lambda\bar{\Lambda}$ pairs. 
The $\Lambda\Lambda$ and $\bar{\Lambda}\bar{\Lambda}$ results are presented as candidates for preliminary status. 
The $\Lambda\bar{\Lambda}$ results were previously approved as a physics preliminary plot.  
The motivation for $\Lambda$ baryon femtoscopy is to complement previous studies, as well as to probe the little-measured strong interactions of these baryons.

The primary effect seen in $\Lambda\bar{\Lambda}$ correlations should be strong final state interactions (FSI).  
A suppression of the correlation function at low relative momentum ($k^*$) is expected due to interactions in the baryon-antibaryon annihilation channel.  
Suppressions of this sort have also been seen in $p \bar{p}$ interactions \cite{Gos:2007cj}, though the effects of this channel in charged particle studies can be somewhat obfuscated by Coulomb enhancement effects in the same $k^*$ region.  
Also discussed in this note are the main influences on $\Lambda\Lambda$ correlations --- strong FSI and Fermi-Dirac statistics.  
A long-term goal of this study is to measure characteristics of the interaction potential of the $\Lambda$ baryons.  
More specifically, the scattering length $f_0$ and effective radius $d_0$ of the interaction can be extracted from two-$\Lambda$ correlations. 
It should also be noted that the H-dibaryon may have two $\Lambda$ as one of its decay channels \cite{PhysRevLett.38.195}, though a recent study at ALICE has suggested that a $\Lambda\Lambda$ bound state is unlikely.  
The study of $\Lambda$ momentum correlations at ALICE may some day provide further measurements of this phenomenon, though endeavors in this direction are too preliminary to report upon at this time.

