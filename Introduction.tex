\section{Introduction}


Hanbury-Brown and Twiss first used two-particle interferometry to measure the angular size of the star Sirius \cite{HanburyBrown:1956bqd}.
Later, it was observed that similar techniques could be employed in a laboratory setting to characterize the size of fireballs created in particle collisions \cite{Goldhaber:1960sf}.
This technique of measuring two-particle correlations at low relative momentum is often referred to as HBT.
As HBT was originally used to study photons and later used to look at pions, it often carries a connotation of Bose-Einstein correlation effects.
Modern analyses look at all manner of hadrons, and as such the correlations might be subject to bosonic, fermionic, or non-identical particle statistics.
As all of these analyses focus on measuring very small sources, the more general term \textit{femtoscopy} has come into use.

% chaotic source

Femtoscopic analyses are capable of measuring spatial and temporal characteristics of heavy-ion collisions, with a particular emphasis made on estimating the homogeneity lengths (also called radii) of particle-emitting sources.  
These analyses traditionally study pions \cite{Goldhaber:1960sf,Aamodt:2011mr} because of their ready availability and sensitivity to the spatial scale of the source.  
However, measurements of heavier particles such as kaons \cite{Abelev:2012ms} and baryons \cite{Gos:2007cj} can serve to complement the pion results.  
One motivation for studying an assortment of heavier particles is to test the hydrodynamic prediction that radial flow should cause the source radii to scale with the transverse mass $m_{\mathrm{T}}$ of the particles \cite{Csorgo:1995bi,Lisa:2005dd}.

Baryon--(anti)baryon correlation functions are also useful in the study of final state interactions (FSI).  
Measurements of scattering lengths of pairs such as p$\Lambda$ and $\Lambda\Lambda$ are of interest for rescattering calculations, and, in the latter case, for understanding the properties of neutron stars \cite{SchaffnerBielich:2008kb,Wang:2010gr}.  
Baryon-antibaryon correlations allow for the study of annihilation processes.  

In this note, we present two-$\Lambda$ momentum correlation functions measured in PbPb collisions at $\sqrt{s_{NN}}=2.76$ TeV by the ALICE Collaboration.  
Results are shown for $\Lambda\Lambda$, $\bar{\Lambda}\bar{\Lambda}$ and $\Lambda\bar{\Lambda}$ pairs. 
The $\Lambda\Lambda$ and $\bar{\Lambda}\bar{\Lambda}$ results are presented as candidates for preliminary status. 
The $\Lambda\bar{\Lambda}$ results were previously approved as a physics preliminary plot.  
The motivation for $\Lambda$ baryon femtoscopy is to complement previous studies, as well as to probe the little-measured strong interactions of these baryons.

The primary effect seen in $\Lambda\bar{\Lambda}$ correlations should be strong final state interactions (FSI).  
A suppression of the correlation function at low relative momentum ($k^*$) is expected due to interactions in the baryon-antibaryon annihilation channel.  
Suppressions of this sort have also been seen in $p \bar{p}$ interactions \cite{Gos:2007cj}, though the effects of this channel in charged particle studies can be somewhat obfuscated by Coulomb enhancement effects in the same $k^*$ region.  
Also discussed in this note are the main influences on $\Lambda\Lambda$ correlations --- strong FSI and Fermi-Dirac statistics.  
A long-term goal of this study is to measure characteristics of the interaction potential of the $\Lambda$ baryons.  
More specifically, the scattering length $f_0$ and effective radius $d_0$ of the interaction can be extracted from two-$\Lambda$ correlations. 
It should also be noted that the H-dibaryon may have two $\Lambda$ as one of its decay channels \cite{PhysRevLett.38.195}, though a recent study at ALICE has suggested that a $\Lambda\Lambda$ bound state is unlikely.  
The study of $\Lambda$ momentum correlations at ALICE may some day provide further measurements of this phenomenon, though endeavors in this direction are too preliminary to report upon at this time.