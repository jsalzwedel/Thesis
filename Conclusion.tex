\section{Conclusion}
\label{sec:Conclusion}

In this thesis...

% Summary of experimental steps
% Summary of theoretical steps
% List results, and include some interpretation (QS, annihilation)

Results have been shown for correlation functions of $\Lambda\Lambda$, $\bar{\Lambda}\bar{\Lambda}$ and $\Lambda\bar{\Lambda}$ pairs.  
The low- to mid-$k^*$ suppression of the $\Lambda\bar{\Lambda}$ correlations in each of the three centrality bins may indicate pair annihilation processes reminiscent of those reported in other studies.


% Talk about the results

% Radius results

The measured radii are unintuitive in a couple ways.
If one considers the two-particle emission functions to be simple convolutions of the single-particle $\Lambda$ and $\bar{\Lambda}$ emission functions, one might assume that $\Lambda\Lambda$, $\bar{\Lambda}\bar{\Lambda}$, and $\Lambda\bar{\Lambda}$ would all have the same radius.
Instead, we see that the $\Lambda\Lambda\oplus\bar{\Lambda}\bar{\Lambda}$ radii are larger than the comparable centrality $\Lambda\bar{\Lambda}$ radii by almost a factor of two.
Another peculiarity is that the $\Lambda\Lambda\oplus\bar{\Lambda}\bar{\Lambda}$ radius is much larger than what one might expect from $m_\mathrm{T}$ scaling of the $\mathrm{pp}$, $\mathrm{\bar{pp}}$, and $\mathrm{K^0_s}\mathrm{K^0_s}$ radii.


% unclear background treatment. Some experiments have alleviated some of the background effects incorporating event-plane binning into their correlation function construction.

Anomalous differences between $\Lambda\Lambda\oplus\bar{\Lambda}\bar{\Lambda}$ and $\Lambda\bar{\Lambda}$ radii.
This difference may be due to over- or under-corrected detector effects such as splitting and merging.
There is some correlation between the measured positions of charged tracks and the momenta of those tracks and their parents
Through that correlation, the detector effects of splitting and merging can show up as false signals in relative momentum correlation functions.
It may be the case that true physics correlations can show up as false signals in average separation correlation functions, the purpose of which is to detect and remove splitting and merging.
If true physics effects bleed into the average separation plots, it may lead us to impose strict cuts which unintentionally remove those physics effects from the femtoscopic correlation functions.
Such a cut may lead to spurious fit results, such as the anomalously large $\Lambda\bar{\Lambda}\oplus\bar{\Lambda}\bar{\Lambda}$ radii.
Future analyses may find merit in investigating alternative observables to cut upon.
For example, one might require the decay lengths of the V0s to differ by several centimeters.
In doing so, this would naturally, though indirectly, limit splitting and merging effects.
Meanwhile, the risk of unintentionally cut out true physics results seems small, as it is unlikely that there is any correlation between the relative momentum of particles and their relative decay positions.

Of course, it is also possible that the radius results presented here are correct, and that there is truly a difference between the emission functions of $\Lambda\Lambda$/$\bar{\Lambda}\bar{\Lambda}$ and $\Lambda\bar{\Lambda}$.
One example of this would be if the $\Lambda\bar{\Lambda}$ emission function contains a not-insignificant

For example, near-threshold pair creation of $\Lambda\bar{\Lambda}$ particles could cause low-$k^*$ pairs to be emitted with a small spatial separation.
If there are a sizable number of these pairs, the two-particle emission function would have a component with a small radius, similar to the core-halo effect in pion femtoscopy (in addition to the usual Gaussian signal measured in pion correlation functions, there is also a very narrow enhancement in the lowest few $k^*$ bins arising from a very large emission region, the pions that come from the decays of long-lived resonances).
In the $\Lambda\bar{\Lambda}$ case, the two-particle emission function would have a component with a small radius from the pair-created particles, as well the expected medium-sized radius component arising from the convolution of the two single-particle emission functions.
As the fit function we employ only has one radius parameter, the extracted radius may be something of an average of the size of these two regions.
Because of that, the extracted $\Lambda\bar{\Lambda}$ radius would be smaller than the radius of either of the $\Lambda$ or $\bar{\Lambda}$ single-particle emission functions, and thus smaller than the $\Lambda\Lambda\oplus\bar{\Lambda}\bar{\Lambda}$ radius.

 

% STAR hypothesized that the small pLambdabar radius may have arisen from pair production types of effects..


% Interaction results

This approach may have applicability for measuring the poorly known interactions of other pairs, such as $\Lambda\Xi^-$, $\Lambda K^-$ and $\Lambda K^+$, or $Xi^-K^-$ and $Xi^-K+$.
The $\Lambda K^\pm$ and $\Xi^-K^\pm$ results in particular may be interesting; the kaon charge conjugation may reveal differences in inelastic interactions between pairs containing $s$ and $\bar{s}$ quarks, versus pairs that contain $u$ and $\bar{u}$ quarks or no corresponding quark-antiquark particles at all.

% Joint fitting between pp, pL, LL (and similarly for the corresponding particle-antiparticle pairs) can increase the precision of radius and scattering lengths for these pairs, and it may provide enough constraints to measure the scattering lengths of the residual pairs such as $\Lambda$\Sigma^0$.