\chapter{Conclusion}
\label{sec:Conclusion}

In this thesis, we presented centrality-dependent femtoscopy of $\Lambda$ and $\bar{\Lambda}$ baryons measured in $\sqrt{s_\mathrm{NN}} = 2.76$ TeV Pb-Pb collisions by the ALICE collaboration at the LHC.
This is the first measurement of $\Lambda\bar{\Lambda}$ femtoscopy, and the first measurement of $\Lambda\Lambda\oplus\bar{\Lambda}\bar{\Lambda}$ to account for residual correlations using the same framework that is employed to calculate the primary correlations.

Our correlation function model follows the analytic framework laid out by Lednicky and Lyuboshits.
It utilizes a one-dimensional Gaussian parametrizaton  for the two-particle emission function that describes the relative separation of the emission points of particles.
The particle interactions are accounted for with an s-wave scattering parameterization of the two-particle wave function, including antisymmetrization for the $\Lambda\Lambda$ and $\bar{\Lambda}\bar{\Lambda}$ pairs.
For s-wave scattering, we use the asymptotic form of an incoming plane wave and an outgoing spherical wave, where the scattering amplitude depends on the scattering length $f_0$ and the effective range of interaction $d_0$.
When fitting, we constrained the same scattering parameters across each centrality bin to improve the precision of the measurement.
In general, systematic uncertainties were dominated by the treatment of the non-femtoscopic background at low-$k^*$.

Using Monte Carlo HIJING events with ALICE reconstruction effects, we estimated the number of pairs where a $\Lambda$ came from a $\Sigma$, $\Xi^-$, $\Xi^0$, $\Omega$, or false reconstruction, and we found that only 25\% of pairs have two primary $\Lambda$ and/or $\bar{\Lambda}$.
The rest have at least one secondary $\Lambda$, and the physics describing the contribution of those pairs to the correlation function depends on the interactions between the parent particles.
We estimated their correlation function contributions using the same analytic framework used to calculate the primary $\Lambda\Lambda\oplus\bar{\Lambda}\bar{\Lambda}$ and $\Lambda\bar{\Lambda}$ correlation functions.
In that calculation, we assumed that the parent pairs interacted with the same strength ($f_0$ and $d_0$) and femtoscopic radius ($R$) as primary pairs, then we smeared out the relative momentum dependence according to the parent-to-$\Lambda$ decay kinematics.
We used the combined femtoscopic contribution of each  pair type to fit the measured correlation functions.
In this way, we were able to turn the heavier hyperons' signal contamination into additional statistical power.

In this analysis, we extracted femtoscopic radii for three centralities of $\Lambda\bar{\Lambda}$ and two centralities of $\Lambda\Lambda\oplus\bar{\Lambda}\bar{\Lambda}$.
Hydrodynamic calculations suggest that there should be a rough scaling of the radii with the inverse of the transverse mass of the particles.
The radii we have measured serve to complement other femtoscopic analyses by providing measurements at large transverse mass $m_\mathrm{T}$.
For $\Lambda\bar{\Lambda}$, we measured $R_{\Lambda\bar{\Lambda},0-10\%} = 2.9\pm 0.3 \pm 0.3$ fm, $R_{\Lambda\bar{\Lambda},10-30\%} = 2.3\pm 0.2 \pm 0.2$ fm, and $R_{\Lambda\bar{\Lambda},30-50\%} = 1.8\pm 0.2 \pm 0.2$ fm. 
While these points do follow the trend set by pions, kaons, and protons, the $\Lambda\Lambda\oplus\bar{\Lambda}\bar{\Lambda}$ break this scaling dramatically.
Here we measured $R_{\Lambda\Lambda\oplus\bar{\Lambda}\bar{\Lambda}, 0-10\%} = 5.4 \pm 0.6 \pm 0.2$ fm and  $R_{\Lambda\Lambda\oplus\bar{\Lambda}\bar{\Lambda}, 10-30\%} = 4.2 \pm 0.5 \pm 0.2$ fm.
We did not have sufficient statistics to fit the 30-50\% centrality bin.
These $\Lambda\Lambda\oplus\bar{\Lambda}\bar{\Lambda}$ radii are nearly double their $\Lambda\bar{\Lambda}$ counterparts.
It is unclear at this time what physics effects might lead to this separation.
Pair creation might lead to $\Lambda\bar{\Lambda}$ radii being smaller than $\Lambda\Lambda\oplus\bar{\Lambda}\bar{\Lambda}$ radii, but would not account for $\Lambda\Lambda\oplus\bar{\Lambda}\bar{\Lambda}$ radii deviating significantly from the $m_\mathrm{T}$ trends.
Another possibility is that the $\Lambda\Lambda\oplus\bar{\Lambda}\bar{\Lambda}$ correlation function, which only extends through a few $k^*$ bins, might have under or overcompensated detector effects that have not been included within the systematic errors.

We measured the scattering parameters of these systems. This information can help constrain hadron cascade models, hypernuclear structure calculations, and neutron star equations of state.
We found that the real part of the scattering length $\Re f_{0,\Lambda\Lambda\oplus\bar{\Lambda}\bar{\Lambda}} = -0.6 \pm 0.3 \pm 0.05$ fm, which is within the systematic uncertainty of STAR's measurement.
This interaction is small, about an order of magnitude smaller than the deuteron bound state, which supports claims that $\Lambda\Lambda$ has no dibaryon bound state.
The effect of the nuclear final state interactions on the $\Lambda\Lambda$ pair appears to be small compared to the contribution from Fermi-Dirac suppression.
The scattering length measured by femtoscopy experiments is approximately the same magnitude as the estimate arising from hypernuclear structure experiments, but they are opposite in sign.
It is unclear how to rectify these differences at this time.
The real part of the scattering lengths of $\Lambda\bar{\Lambda}$ and $\Lambda\Lambda\oplus\bar{\Lambda}\bar{\Lambda}$ are comparable in size ($\Re f_{0,\Lambda\bar{\Lambda}} = -0.5 \pm 0.1 \pm 0.1$ fm), suggesting that the elastic interactions of these hyperons behave the same.

The imaginary part of the scattering length $\Im f_0$ accounts for the inelastic process of annihilation.
As a baryon-baryon pair, $\Lambda\Lambda\oplus\bar{\Lambda}\bar{\Lambda}$ has no imaginary component to its scattering length, but we have measured it for $\Lambda\bar{\Lambda}$.
Here, $\Im f_{0,\Lambda\bar{\Lambda}} = 0.14 \pm 0.09 \pm 0.2$ fm.
This is considerably smaller than the measured values for p$\mathrm{p}$ and p$\bar{\Lambda}$, both of which appear to fall around 0.8 fm.

The effective range of interaction $d_0$ is a correction to the scattering amplitude, and its effect on the correlation function is small compared to that of the scattering length.
Given that and the small FSI seen in $\Lambda\Lambda\oplus\bar{\Lambda}\bar{\Lambda}$, it is not surprising that $d_0$ is not particularly well constrained for $\Lambda\Lambda\oplus\bar{\Lambda}\bar{\Lambda}$, with $d_{0,\Lambda\Lambda\oplus\bar{\Lambda}\bar{\Lambda}} = 5.3 \pm 2.7 \pm 0.3$ fm.
The $\Lambda\bar{\Lambda}$ FSI effects extend to a much larger $k^*$ range, so we have better resolving power for the effective range $d_{0,\Lambda\bar{\Lambda}} = 1.7 \pm 0.2 \pm 0.2$ fm.

Efforts are currently underway within ALICE to perform joint fitting of $\Lambda\Lambda\oplus\bar{\Lambda}\bar{\Lambda}$ ($\Lambda\bar{\Lambda}$) with other baryon-baryon (baryon-antibaryon) systems.
The femtoscopic analyses of pp and p$\Lambda$ have residual correlation effects from $\Lambda\Lambda$, as well all the heavier pairs such as $\Lambda\Sigma$ and $\Lambda\Xi$.
The underlying correlations from these parent pairs should be the same, regardless of the type of the primary correlation that is being measured.
Whether it feeds down into pp, p$\Lambda$, or $\Lambda\Lambda$, the $\Lambda\Sigma$ correlation has one scattering length and one femtoscopic radius at a given event centrality and $k_\mathrm{T}$.
Likewise, $\Lambda\Lambda$ correlation function is the same whether it is a primary correlation as seen in this analysis, or a residual correlation in a p$\Lambda$ analysis.
Joint fitting---simultaneous $\chi^2$ fitting of pp, p$\Lambda$, and $\Lambda\Lambda$ using shared scattering lengths and radii amongst the primary and residual correlations---has the potential to provide significantly stronger constraints on the interactions and radii of $\Lambda\Lambda$ than a solo fit.

We have used femtoscopy with residual correlation contributions to measure the interactions of $\Lambda\Lambda$ and $\Lambda\bar{\Lambda}$.
Residual correlation contributions appear to be much larger in baryon-antibaryon pairs.
Despite this, the measured scattering parameters and radii for $\Lambda\bar{\Lambda}$ appear to be robust.
Nonetheless, some mysteries remain regarding both the $\Lambda\Lambda\oplus\bar{\Lambda}\bar{\Lambda}$ radii and scattering length.
These may be resolved with further investigation of methods to remove detector effects.
If issues such as these can be resolved, this approach may have applicability for measuring the poorly known interactions of other pairs, such as $\Lambda\Xi^-$, $\Lambda K^-$ and $\Xi^-K^-$.
The Pb-Pb data taken during the LHC's second run period (2015-2018) will afford a chance to examine these phenomena again at a higher energy, $\sqrt{s_\mathrm{NN}} = 5.02$ TeV.


% Joint fitting between pp, pL, LL (and similarly for the corresponding particle-antiparticle pairs) can increase the precision of radius and scattering lengths for these pairs, and it may provide enough constraints to measure the scattering lengths of the residual pairs such as $\Lambda$\Sigma^0$.