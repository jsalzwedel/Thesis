\subsection{Momentum resolution corrections}
\label{sec:MomResCorrectFit}

Momentum resolution correction info...

Using a HIJING MC data set, we created a two-dimensional histogram of $k^*_\mathrm{true}$ vs $k^*_\mathrm{recon}$ for pairs of $\Lambda$.
We removed fake $\Lambda$ from the analysis.
However, because momentum resolution smearing affects the daughters of both primary and secondary $\Lambda$, and because we were in need of a large number of pairs, we did not distinguish between primary and secondary $\Lambda$.
We used mixed event pairs rather than same event pairs to ensure we had sufficient statistics.

% Figure for LL
% Figure for AA?
% Figure for LA

% Show figures with and without log-scale

Figures \ref{fig:MomResLL} show the possible phase space for how relative momentum can be smeared.
Each row of each figure is normalized separately.
Given a pair with a particular true $k^*$ value, the likelihood of that pair having a particular reconstructed $k^*$ value is given by the weights in the associated bin of the histogram.

It is important to note that these histograms describe how a $k^*_\mathrm{true}$ distribution can be smeared into a $k^*_\mathrm{recon}$ distribution, but not the reverse process.
The reverse process (unsmearing), would be described by the matrix inverse of these histograms.
Unfortunately, attempts to invert the matrices yield significant aliasing, and the results cannot be used.

However, these momentum resolution matrices can be used to smear the theoretical prediction (i.e. the fit) from $k^*_\mathrm{true}$--space into $k^*_\mathrm{recon}$--space using exactly the same procedure described above for residual correlation smearing.