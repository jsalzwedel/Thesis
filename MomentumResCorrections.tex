\subsection{Momentum resolution corrections}
\label{sec:MomResCorrectFit}

Momentum resolution correction info...

Using a HIJING MC data set, we created a two-dimensional histogram of $k^*_\mathrm{gen}$ vs $k^*_\mathrm{recon}$ for pairs of $\Lambda$.
We removed fake $\Lambda$ from the analysis.
However, because momentum resolution smearing affects the daughters of both primary and secondary $\Lambda$, and because we were in need of a large number of pairs, we did not distinguish between primary and secondary $\Lambda$.
We used mixed event pairs rather than same event pairs to ensure we had sufficient statistics.

% Figure for LL
% Figure for AA?
% Figure for LA

% Show figures with and without log-scale

Figures \ref{fig:MomResLL} show the possible phase space for how relative momentum can be smeared.
Each row of each figure is normalized separately.
Given a pair with a particular true $k^*$ value, the likelihood of that pair having a particular reconstructed $k^*$ value is given by the weights in the associated bin of the histogram.

It is important to note that these histograms describe how a $k^*_\mathrm{gen}$ distribution can be smeared into a $k^*_\mathrm{recon}$ distribution, but not the reverse process (unsmearing).
Unsmearing would be described by the matrix inverse of these histograms.
Unfortunately, attempts to invert the matrices yield significant aliasing, and the results cannot be used.

That said, these momentum resolution matrices can be used to smear the theoretical prediction (i.e. the fit) from $k^*_\mathrm{gen}$--space into $k^*_\mathrm{recon}$--space using exactly the same procedure described above for residual correlation smearing:
\begin{equation}
\label{eq:MomResSmear}
C(k^*_{\mathrm{rec}}) \equiv \frac{\displaystyle\sum\limits_{k^*_{\mathrm{gen}}}M(k^*_{\mathrm{rec}},k^*_{\mathrm{gen}})C(k^*_{\mathrm{gen}})}{\displaystyle\sum\limits_{k^*_{\mathrm{gen}}}M(k^*_{\mathrm{rec}},k^*_{\mathrm{gen}})}.
\end{equation} 
Here, $M(k^*_{\mathrm{rec}},k^*_{\mathrm{gen}})$ is the momentum resolution smearing matrix.
One can perform this transformation on the fit function to ensure that both the data and the fit have the same smearing. 

There are two main benefits to treating the momentum resolution effects in this way. 
First, it is no longer necessary to run an iterative analysis (see Section \ref{sec:MomentumResCorrectionCF}). 
This method requires a one--time analysis of Monte Carlo data to produce the momentum resolution smearing matrix. 
Second, the resolution correction can be incorporated directly into the fit procedure as an intermediate step between calculating the fit function and taking the $\chi^2$ difference with the data.

Momentum resolution affects daughter tracks, and as such the primary or secondary nature of the parent $\Lambda$ is irrelevant.
Therefore, when smearing the fit function, momentum resolution smearing should be applied after residual correlation smearing. 
In fact, as both types of smearing are effectively forms of matrix algebra, the two smearing effects can be rolled together. 
In that case, $C(k^*_\mathrm{gen})$ in Eqn.\ \ref{eq:MomResSmear} is the theoretical correlation function for some pair type expressed in terms of $k^*_{\Lambda\Lambda}$, as given by Eqn.\ \ref{eq:ResCorSmear}.

Combined together, they give (in the case of $\Lambda\Sigma$ feeddown)
\begin{equation}
C_{\Lambda\Sigma}(k^{*\Lambda\Lambda}_{\mathrm{rec}}) \equiv \frac{\displaystyle\sum\limits_{k^{*\Lambda\Lambda}_{\mathrm{gen}}}M(k^{*\Lambda\Lambda}_{\mathrm{rec}},k^{*\Lambda\Lambda}_{\mathrm{gen}})
 \frac{\displaystyle\sum\limits_{k^{*\Lambda\Sigma}_{\mathrm{gen}}}T(k^{*\Lambda\Lambda}_\mathrm{gen},k^{*\Lambda\Sigma}_{\mathrm{gen}})C_{\Lambda\Sigma}(k^{*\Lambda\Sigma}_{\mathrm{gen}})}{\displaystyle\sum\limits_{k^{*\Lambda\Sigma}_{\mathrm{gen}}}T(k^{*\Lambda\Lambda}_\mathrm{gen},k^{*\Lambda\Sigma}_{\mathrm{gen}})}
}{\displaystyle\sum\limits_{k^{*\Lambda\Lambda}_{\mathrm{gen}}}M(k^{*\Lambda\Lambda}_{\mathrm{rec}},k^{*\Lambda\Lambda}_{\mathrm{gen}})}.
\end{equation}

This becomes much cleaner if we introduce the partially-normalized (i.e.\ along one axis) matrices
\begin{equation}
\tilde{M}(k^*_{\mathrm{rec}},k^*_{\mathrm{gen}}) \equiv \frac{M(k^*_{\mathrm{rec}},k^*_{\mathrm{gen}})}{\displaystyle\sum\limits_{k^*_{\mathrm{gen}}}M(k^*_{\mathrm{rec}},k^*_{\mathrm{gen}})}
\end{equation}
and
\begin{equation}
\tilde{T}(k^*_{\Lambda\Lambda},k^*_{\Lambda\Sigma}) \equiv \frac{T(k^*_{\Lambda\Lambda},k^*_{\Lambda\Sigma})}{\displaystyle\sum\limits_{k^*_{\Lambda\Sigma}}T(k^*_{\Lambda\Lambda},k^*_{\Lambda\Sigma})}.
\end{equation}
Then we get the more compact
\begin{equation}
C_{\Lambda\Sigma}(k^{*\Lambda\Lambda}_{\mathrm{rec}}) = \displaystyle\sum\limits_{k^{*\Lambda\Lambda}_{\mathrm{gen}}}\tilde{M}(k^{*\Lambda\Lambda}_{\mathrm{rec}},k^{*\Lambda\Lambda}_{\mathrm{gen}})\displaystyle\sum\limits_{k^{*\Lambda\Sigma}_{\mathrm{gen}}}\tilde{T}(k^{*\Lambda\Lambda}_\mathrm{gen},k^{*\Lambda\Sigma}_{\mathrm{gen}})C_{\Lambda\Sigma}(k^{*\Lambda\Sigma}_{\mathrm{gen}})
\end{equation}


Note that as there are multiple residual correlation transform matrices (one for each pair that decays into $\Lambda\Lambda$, there are an equal number of full smearing matrices.
For consistency, we treat the momentum resolution smearing of primary $\Lambda\Lambda$ as a full smearing matrix, though here the residual correlation matrix would be the identity matrix.

 