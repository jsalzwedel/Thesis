\section{Correlation function model}
\label{sec:CorrelationFunctionModel}

% introduce section
In this section, we'll walk through the theoretical framework for analyzing femtoscopic correlations.
We'll start with a general description of a two-particle correlation function via the Koonin-Pratt paradigm \cite{Koonin:1977fh, Pratt:1990zq, Lisa:2005dd}.
From there, we will discuss several ways to write the source function which characterizes the size and shape of the interaction region, and we will give extra attention to the spherically Gaussian source employed in this analysis.
After that, we will describe the two-particle wave function that accounts for potentially spin-dependent interactions via low-energy scattering theory.
We will use the chosen source function and wave function to derive an analytic form for the one-dimensional correlation function.
Finally, we will discuss the observables that are accessible through this model.
 
\subsection{Koonin-Pratt equation}
\label{sec:KooninPratt}

As discussed in Section \ref{sec:CFconstruct}, correlation functions are constructed in such a way as to divide out uncorrelated phase-space effects and therefore to leave behind only the correlated part of the two-particle phase space.
We assume that the uncorrelated two-particle phase space is just a product of two single-particle phase spaces.
The single-particle emission function $S(x,p)$ (also referred to as the single-particle phase space or Wigner density) is related to the single-particle momentum distribution function of the species:
\begin{equation}
\label{eq:SingleParticlePhaseSpace}
E_p\frac{dN}{d^3p} = \int \mathop{d^4x} S(x,p).
\end{equation}
The two-particle correlation function can be expressed as a ratio of distributions:
\begin{equation}
\label{eq:YieldsCF}
C(p_1,p_2) = \frac{E_1E_2\frac{dN}{\mathop{d^3p_1}\mathop{d^3p_2}}}{(E_1\frac{dN}{\mathop{d^3p_1}})(E_2\frac{dN}{\mathop{d^3p_2}})},
\end{equation}
where the numerator term is the two-particle momentum distribution, i.e.\ the distribution of pairs of particles with momenta $p_1$ and $p_2$.

We can similarly write the correlation function as a ratio of emission functions.
Here, the denominator is just a product of two single-particle emissions.
Assuming the particles are created and emitted independently, the numerator is likewise a product of two single-particle emission functions, but we also include the two-particle wave function $\Psi(r,k)$ which introduces correlations between particles.
\begin{equation}
\label{eq:BiggerKooninPratt}
C(p_1,p_2) = \frac{\int d^4x_1d^4x_2 S_1(x_1,p_1) S_2(x_2,p_2)\abs{\Psi(r,k^*)}^2}{\int d^4x_1d^4x_2 S_1(x_1,p_1) S_2(x_2,p_2)}.
\end{equation}
The wave function only cares about the relative separation $r^\mu = x_1^\mu - x_2^\mu$ and the relative momentum $k^\mu$ of the particles. 
The four-vector momentum difference $k^\mu$ and its corresponding invariant relative momentum $k^*$ are defined in Section \ref{sec:RelativeMomentum}.

As the wave function only cares about relative quantities, let us change variables.
Let $p_1^\mu = P^\mu/2 + k^\mu$ and $p_2^\mu = P^\mu/2 - k^\mu$, where $P^\mu = p_1^\mu + p_2^\mu$.
With that, we can define a two-particle emission function
\begin{equation}
\label{eq:TwoParticleEmissionFunction}
S(r, P/2 + k)=\frac{\int\mathop{d^4x_1}\mathop{d^4x_2}S_1(x_1,p_1) S_2(x_2,p_2)\delta(r - x_1 +x_2)}{\int d^4x_1d^4x_2 S_1(x_1,p_1) S_2(x_2,p_2)}.
\end{equation}
We assume that for a given total momentum $P$ and small relative momentum $k*$ (i.e. for particles with approximately the same momentum), the emission function is independent of the momentum difference
\begin{equation}
S(r,P/2+k) \approx S(r,P).
\end{equation}
There is one last approximation. It is common to speak of the emission function for a given total momentum $P$ or transverse momentum $k_T = p_{1,T} + p_{2,T}$.
In principle, correlation functions can be created using infinitesimally small momentum bins, in which case a source function $S_P(r)$ can be characterized for each of those $P$ bins.
In practice, the size of the $P$ or $k_T$ bins is chosen based on the statistical limitations of the experiment; one uses as many bins as one can while still getting sufficient statistics.
In any case, it is common to assume that the emission function is independent of momentum within each measured $P$ bin.
Because the emission function describes the region where particles have approximately the same momentum, that region is called the region of homogeneity \cite{Akkelin:1995gh}.

% Implication: When we measure the source function, we are measuring the size/shape of the region where particles have the same momentum

With these simplifications, we can rewrite Eqn.\ \ref{eq:BiggerKooninPratt} as 
\begin{equation}
\label{eq:KooninPratt}
C_P(k^*) =  \int d^3r S_{P} (r) \abs{\Psi(k^*,r)}^2.
\end{equation}
Equations \ref{eq:BiggerKooninPratt} and \ref{eq:KooninPratt} are both called the Koonin-Pratt equation. 
In the following sections, we'll discuss the two-particle emission function $S(r)$ that we employ to characterize the interaction region, as well as a general wave function that can describe interactions between both identical and non-identical particle pairs in the absence of Coulomb effects.

\subsection{Source functions}
\label{sec:SourceFunctions}

As discussed above and in Section \ref{sec:FemtoSources}, the two-particle emission function describes the size and shape of the interaction region.
However, the size of the source is not the size of the entire fireball at freezeout.
Instead, it is the size of the region of homogeneity, the region where particles with total momentum $\vec{P}$ are emitted with the same momentum (and therefore $k^* = 0$).
%This is illustrated by the cartoon in Figure \ref{fig:HomogeneityRegion}.

In practice, there are a number of ways to characterize the source function.
One of the simplest treatments is to assume that the source is described by a spherically symmetric function such as a Gaussian.
One can then write the source with a single, one-dimensional parameter, often a Gaussian radius $R$.
There are several weaknesses to this method.
For example the source is likely not spherically symmetric --- at the moment of the collision of the ions the interaction region roughly has the shape of a pancake or an almond,
% (see Figure \ref{fig:PancakeAlmond})
and it need not expand isotropically in the longitudinal and transverse directions.
Additionally, a Gaussian may not always be the best approximation; sometimes an exponential or an Edgeworth expansion may work better \cite{Abelev:2014pja}.
Despite these flaws, the one-dimensional approach has merit for analyses with limited statistics, since there is less risk of overfitting the data with too many parameters.
And as we'll see below, there is an analytic form for the one-dimensional correlation function, which is not necessarily the case for multi-dimensional analyses.
Because $\Lambda$ yields are much smaller than $\pi$ yields, and the $\Lambda\Lambda$ signal in particular is quite weak (see Figure \ref{fig:CFLamLamALamALam010}), this study employs the 1D approach.
We assume a normalized, one-dimensional, spherically gaussian source of width $R$,
\begin{equation}
\label{eq:Gaussian1DSource}
S_P(r) \sim \exp{(-\frac{r^2}{4R^2})}.
\end{equation}

Though it is outside the scope of this analysis, it is worth mentioning the multidimensional fitting methods.
There are two common approaches.
One is the out-side-long prescription.
For a detailed description, see \cite{Lisa:2005dd}.
Here, one decomposes both the source function and the relative momentum $k^*$ into three orthogonal directions.
One then calculates the relative momentum of the pair in the longitudinally comoving system (LCMS).
The longitudinal direction is defined to point along the beam axis (i.e. the axis along with the ions travel on their collision course), so in the LCMS frame the pair has net longitudinal momentum $P_\mathrm{z} = 0$.
The out direction points along the total momentum of the pair $\vec{P}=\vec{P}_{\mathrm{T}}$, and the side direction is perpendicular to the other two.
Here, one usually assumes a Gaussian approximation for each direction with radii $R_\mathrm{out}$, $R_\mathrm{side}$, and $R_\mathrm{long}$.
In some cases, there may also be cross terms such as $R^2_{\mathrm{out,side}}$, as well as effects arising from differences in the particles' emission times.

The other multidimensional approach involves decomposing the 3D correlation function $C(k^*)$ in terms of spherical harmonics $C_{l,m}(\abs{k^*},\theta, \phi)$ \cite{Chajecki:2008vg}.
This has several advantages over the out-side-long method.
For one, it is straightforward to plot a particular spherical component of the correlation function $C_{l,m}$, whereas there are ambiguities when making one-dimensional projections of $C(k^*)$.
For another, various physical symmetries should cause certain components of $C_{l,m}$ to be zero.
If they do not vanish, it usually a symptom that something is procedurally wrong with the analysis.
Therefore, inspection of the spherical components can help with analysis debugging.
Given the strong signal seen in this analysis's $\Lambda\bar{\Lambda}$ correlation functions (see Figure \ref{fig:CFLamALam010}), said correlation functions may be a good candidate for spherical harmonic decomposition in a future study.




% In some cases, inverted to get source function
 
\subsection{Wave function}
\label{sec:WaveFunction}



%Introduce section: As mentioned above, we 

For identical particles, we can write the asymptotic solution to the non-symmatrized two-particle wave function $\psi$ as a superposition of an incoming plane-wave and an outgoing spherical wave
\begin{equation}
\label{eq:swaveScattering}
\psi_{p_1p_2}(r,k^*) = e^{ik^*r} + (1 + (-1)^S)f(k^*)\frac{e^{ik^*r}}{r},
\end{equation}
Here 
\begin{equation}
\label{eq:ScatteringAmplitude}
f(k^*) = \left(\frac{1}{f_0} + \frac{1}{2}d_0k^{*2}-ik^*\right)^{-1}
\end{equation}
is the s-wave scattering amplitude in the effective-range approximation \cite{LANDAU1977502}, $f_0$ is the complex scattering length which characterizes the strength of the interaction, $d_0$ is the effective range of interaction, and $S$ is the total spin of the pair.

For non-identical particles, the full wave function $\Psi(r,k^*) = \psi_{p_1p_2}(r,k^*)$, and there is no $(1 + (-1)^S)$ prefactor on the scattering amplitude. 
For identical, unpolarized, spin-$\frac{1}{2}$ fermions, the symmetrized wave function is given by 
\begin{equation}
\label{eq:SymmetrizedWaveFunction}
\abs{\Psi(r,k^*)}^2 = \frac{1}{2}\left(\frac{3}{4} \abs{\psi_{p_1p_2}(r,k^*) - \psi_{p_2p_1}(r,k^*)}^2
+ \frac{1}{4} \abs{\psi_{p_1p_2}(r,k^*) + \psi_{p_2p_1}(r,k^*)}^2\right),
\end{equation}
where $\frac{3}{4}$ of the pairs are in the symmetric spin-triplet state and antisymmetric spatial state, and the other $\frac{1}{4}$ are in the antisymmetric spin-singlet state and symmetric spatial state.
In principle, the scattering length $f_0$ can be a spin-dependent quantity, as is the case for the p$\Lambda$ interaction \cite{Adams:2005ws}.

Substituting in Eqn.\ \ref{eq:swaveScattering} and noting that $S=1$ for the spin triplet state and $S=0$ for the spin singlet state, we find
\begin{equation}
\label{eq:IdenticalUnpolarizedFermionicWaveFunction}
\abs{\Psi(r,k^*)}^2 = \frac{1}{2}\left(\frac{3}{4} \abs{e^{ik^*r} - e^{-ik^*r}}^2
+ \frac{1}{4} \abs{e^{ik^*r} + f(k^*)\frac{e^{ik^*r}}{r} + e^{-ik^*r} + f(k^*)\frac{e^{-ik^*r}}{r}}^2\right).
\end{equation}
%\begin{equation}
%\abs{\Psi(r,k^*)}^2 =\frac{1}{2}\left(3\sin^2(k^*r)
%+ \cos^2(k^*r)(1+2 \Re(\frac{f}{r}) + \frac{\abs{f}^2}{r})  \right).
%\end{equation}

%Then (fill in a general derivation of the Lednicky eqn  \cite{lednicky82}) ...




\subsection{1D analytic model}
\label{sec:AnalyticModel}

% Integrate over the source
We then evaluate the Koonin-Pratt equation (Eqn.\ \ref{eq:KooninPratt}) with the use of our two-particle source function (Eqn.\ \ref{eq:Gaussian1DSource}) and symmetrized wave function (Eqn.\ \ref{eq:IdenticalUnpolarizedFermionicWaveFunction}).
The details of the calculation can be found in \cite{lednicky82}.
The resulting analytic, one-dimensional, femtoscopic correlation function is called the Lednicky and Lyuboshits equation (abridged to Lednicky equation), which can be broken down as
\begin{equation}
\label{eq:GeneralCorrelationFunction}
C(k^*)= 1 + C_{\mathrm{QS}}(k^*)+C_{\mathrm{FSI}}(k^*).
\end{equation}
$C_{\mathrm{QS}}$ describes plane-wave quantum interference:
\begin{equation}
C_{\mathrm{QS}}(k^*) = \alpha e^{-4k^{*2}R^2}
\end{equation}
where $R$ is the approximate 1D spherical source size, and $\alpha = (-1)^{2j}/(2j+1)$ for identical particles with spins $j$, and $\alpha = 0$ for non-identical particles.
$C_{\mathrm{FSI}}$ describes the s-wave strong final state interaction of the particles:
\begin{equation}
\label{eq:Lednicky}
C_{\mathrm{FSI}}(k^*)= (1+\alpha)\left[\frac{1}{2}\abs{\frac{f(k^*)}{R}}^2(1-\frac{d_0}{2\sqrt{\pi}R})+\frac{2\Re f(k^*)}{\sqrt{\pi}R}F_1(2k^*R)-\frac{\Im f(k^*)}{R}F_2(2k^*R)\right]
%C_2(k^*)= 1+ \displaystyle\sum\limits_{S}\rho_S\left[\frac{1}{2}\abs{\frac{f^S(k^*)}{R}}^2(1-\frac{d_0^S}{2\sqrt{\pi}R})+\frac{2\Re f^S(k^*)}{\sqrt{\pi}R}F_1(2k^*R)-\frac{\Im f^S(k^*)}{R}F_2(2k^*R)\right]
\end{equation}
where $F_1(z) = \int_0^z \! \mathrm{d}x \, e^{x^2-z^2}/z$, and $F_2(z) = (1-e^{-z^2})/z$.
The $(1+\alpha)$ pre-factor accounts for the relevant (non/anti-)symmetrizaton effects.
The scattering amplitude $f$ (Eqn.\ \ref{eq:ScatteringAmplitude}) is dependent upon the effective range of interaction $d_0$, as well as the complex scattering length $f_0$.  
The real part of the scattering amplitude can contribute either a positive or a negative correlation, but either way the effect is relatively narrow in $k^*$ (on the order of about one hundred MeV/$c$).  
The imaginary part of the scattering amplitude accounts for inelastic processes of baryon-antibaryon annihilation.  
Introducing a non-zero imaginary part to the scattering length produces a wide (hundreds of MeV/$c$) negative correlation.  
%For charged particles, an additional factor \cite{Aamodt:2011kd} is necessary to account for the Coulomb interaction.

Technically, the scattering length and effective range are spin dependent.  
It should be noted that in the case of identical fermions, "the contributions of s-wave interaction of identical nucleons to the correlation function goes to zero [for summary spin state S = 1] (identical nucleons with parallel spins cannot be in the s-wave state)" \cite{lednicky82}.  
As a result, only the spin singlet (i.e. antisymmetric spinor, symmetric spatial wave function) scattering parameters are measured for identical fermions.  
Such is the case for the $\Lambda\Lambda$ and $\bar{\Lambda}\bar{\Lambda}$ correlation measurements.  
However, for non-identical particles such as $\Lambda\bar{\Lambda}$, the factor $(1+\alpha)$ could be replaced with a weighted sum of $C_2$ calculated using separate spin-dependent scattering parameters.  
That calculation would use $\rho_S$, the fraction of pairs in each total spin state S, as the weight.  
For statistical reasons, this analysis will eschew spin-dependent measurements of the $\Lambda\bar{\Lambda}$ scattering parameters, and instead attempt to fit for a spin-averaged value.


\subsection{Observables}
\label{sec:Observables}
For analyses such as proton-proton or proton-neutron, the scattering lengths $f_0$ and effective range of interaction $d_0$ are well specified by years of scattering data, and they can be fixed to the known values.  
In the case of lambda-(anti-)lambda femtoscopy, the interaction is heretofore largely unstudied.  
For this analysis, the scattering lengths and effective range are left as free parameters.  
Many analyses treat $\lambda$, a rough measure of the pair purity, as another free parameter.  
For example, $\lambda_{\Lambda\Lambda}$ would correspond to the ratio of number of true primary $\Lambda\Lambda$ pairs to the total number of pairs in the correlation function. 
The remaining fraction is comprised of all the other combinations of pair types, such as pairs where one $\Lambda$ is real and primary and the other is fake, or pairs between primary and secondary lambdas.  
Our implementation of $\lambda$ parameters is described in Section \ref{sec:TransformedResiduals}, and estimates of those parameters are given in Section \ref{sec:LambdaParams}
The last fit parameter is $R_{\rm inv}$, which characterizes the size of the one-dimensional two-particle emission function.


