\subsection{Correlation function model}
\label{sec:CorrelationFunctionModel}



One can characterize two-particle correlation functions with the Koonin-Pratt equation
\begin{equation}
\label{eq:KooninPratt}
C(\vec{P}, \vec{k^*}) =  \int d^3r S_{\vec{P}} (\vec{r}) \abs{\Psi(\vec{k^*},\vec{r})}^2
\end{equation}
where $\vec{P} = \vec{p_1} + \vec{p_2}$ is the total momentum of a particle pair, $\vec{r}$ is the relative separation of the particles, $k^*$ is the relative momentum of the particles, $\Psi$ is the two-particle wave function that describes the interaction of the particles, and $S_{\vec{P}} (\vec{r})$ is the two-particle emission function which describes the relative position of particles with total momentum $\vec{P}$ at kinetic-freezeout.

\subsubsection{Source functions}
\label{sec:SourceFunctions}

As discussed in Section \ref{sec:FemtoSources}, the two-particle emission/source function describes the size and shape of the interaction region.
However, the size of the source is not the size of the entire fireball at freezeout.
Instead, it is the size of the region of homogeneity, the region where particles with total momentum $\vec{P}$ are emitted with the same momentum (and therefore $k^* = 0$).
This is illustrated by the cartoon in Figure \ref{fig:HomogeneityRegion}.

In practice, there are a number of ways to characterize the source function.
One of the simplest treatments is to assume that the source is described by a spherically symmetric function such as a Gaussian.
One can then write the source with a single, one-dimensional parameter, often a Guassian radius $R$.
There are several weaknesses to this method.
For example the source is likely not spherically symmetric --- at the moment of the collision of the ions the interaction region roughly has the shape of a pancake or an almond (see Figure \ref{fig:PancakeAlmond}), and it need not expand isotropically in the longitudinal and transverse directions.
Additionally, a Gaussian may not always be the best approximation; sometimes an exponential or an Edgeworth expansion may work better \cite{Abelev:2014pja}.
Despite these flaws, the one-dimensional approach has merit for analyses with limited statistics, since there is less risk of overfitting the data with too many parameters.
And as we'll see below, there is an analytic form for the one-dimensional correlation function, which is not necessarily the case for multi-dimensional analyses.
Because $\Lambda$ yields are much smaller than $\pi$ yields, and the $\Lambda\Lambda$ signal in particular is quite weak (see Figure \ref{fig:CFLamLamALamALam010}), this study employs the 1D approach.

Though it is outside the scope of this analysis, it is worth mentioning the multidimensional fitting methods.
There are two common approaches.
One is the out-side-long prescription.
For a detailed description, see \cite{Lisa:2005dd}.
Here, one decomposes both the source function and the relative momentum $k^*$ into three orthogonal directions.
One then calculates the relative momentum of the pair in the longitudinally comoving system (LCMS).
The longitudinal direction is defined to point along the beam axis (i.e. the axis along with the ions travel on their collision course), so in the LCMS frame the pair has net longitudinal momentum $P_\mathrm{z} = 0$.
The out direction points along the total momentum of the pair $\vec{P}=\vec{P}_{\mathrm{T}}$, and the side direction is perpendicular to the other two.
Here, one usually assumes a Gaussian approximation for each direction with radii $R_\mathrm{out}$, $R_\mathrm{side}$, and $R_\mathrm{long}$.
In some cases, there may also be cross terms such as $R^2_{\mathrm{out,side}}$, as well as effects arising from differences in the particles' emission times.

The other multidimensional approach involves decomposing the 3D correlation function $C(\vec{k}^*)$ in terms of spherical harmonics $C_{l,m}(\abs{k^*},\theta, \phi)$ \cite{Chajecki:2008vg}.
This has several advantages over the out-side-long method.
For one, it is straightforward to plot a particular spherical component of the correlation function $C_{l,m}$, whereas there are ambiguities when making one-dimensional projections of $C(\vec{k}^*)$.
For another, various physical symmetries should cause certain components of $C_{l,m}$ to be zero.
If they do not vanish, it usually a symptom that something is procedurally wrong with the analysis.
Therefore, inspection of the spherical components can help with analysis debugging.
Given the strong signal seen in this analysis's $\Lambda\bar{\Lambda}$ correlation functions (see Figure \ref{fig:CFLamALam010}), said correlation functions may be a good candidate for spherical harmonic decomposition in a future study.






% In some cases, inverted to get source function
 
\subsubsection{1D analytic model}
\label{sec:AnalyticModel}
With the assumption of a spherically gaussian source of width $R$, the 1D femtoscopic correlation functions can be calculated analytically \cite{lednicky82} using 

\begin{equation}
\label{eq:GeneralCorrelationFunction}
C(k^*)= 1 + C_{\mathrm{QS}}(k^*)+C_{\mathrm{FSI}}(k^*).
\end{equation}
$C_{\mathrm{QS}}$ describes plane-wave quantum interference:
\begin{equation}
C_{\mathrm{QS}}(k^*) = \alpha e^{-4k^{*2}R^2}
\end{equation}
where $R$ is the approximate 1D spherical source size, and $\alpha = (-1)^{2j}/(2j+1)$ for identical particles with spins $j$, and $\alpha = 0$ for non-identical particles.
$C_{\mathrm{FSI}}$ describes the s-wave strong final state interaction of the particles:
\begin{equation}
\label{eq:Lednicky}
C_{\mathrm{FSI}}(k^*)= (1+\alpha)\left[\frac{1}{2}\abs{\frac{f(k^*)}{R}}^2(1-\frac{d_0}{2\sqrt{\pi}R})+\frac{2\Re f(k^*)}{\sqrt{\pi}R}F_1(2k^*R)-\frac{\Im f(k^*)}{R}F_2(2k^*R)\right]
%C_2(k^*)= 1+ \displaystyle\sum\limits_{S}\rho_S\left[\frac{1}{2}\abs{\frac{f^S(k^*)}{R}}^2(1-\frac{d_0^S}{2\sqrt{\pi}R})+\frac{2\Re f^S(k^*)}{\sqrt{\pi}R}F_1(2k^*R)-\frac{\Im f^S(k^*)}{R}F_2(2k^*R)\right]
\end{equation}
where $F_1(z) = \int_0^z \! \mathrm{d}x \, e^{x^2-z^2}/z$, and $F_2(z) = (1-e^{-z^2})/z$.
The $(1+\alpha)$ pre-factor accounts for the relevant (non/anti-)symmetrizaton effects.
$f(k^*)=(1/f_0+\frac{1}{2}d_0k^*-ik^*)^{-1}$ is the s-wave scattering amplitude, written using the effective range approximation.
The scattering amplitude is dependent upon the effective range of interaction $d_0$, as well as the complex scattering length $f_0$.  
The real part of the scattering amplitude can contribute either a positive or a negative correlation, but either way the effect is relatively narrow in $k^*$ (on the order of about one hundred MeV/$c$).  
The imaginary part of the scattering amplitude accounts for inelastic processes of baryon-antibaryon annihilation.  
Introducing a non-zero imaginary part to the scattering length produces a wide (hundreds of MeV/$c$) negative correlation.  
For charged particles, an additional factor \cite{Aamodt:2011kd} is necessary to account for the Coulomb interaction.

Technically, the scattering length and effective range are spin dependent.  
It should be noted that in the case of identical fermions, "the contributions of s-wave interaction of identical nucleons to the correlation function goes to zero [for summary spin state S = 1] (identical nucleons with parallel spins cannot be in the s-wave state)" \cite{lednicky82}.  
As a result, only the spin singlet (i.e. antisymmetric spinor, symmetric spatial wave function) scattering parameters are measured for identical fermions.  
Such is the case for the $\Lambda\Lambda$ and $\bar{\Lambda}\bar{\Lambda}$ correlation measurements.  
However, for non-identical particles such as $\Lambda\bar{\Lambda}$, the factor $(1+\alpha)$ could be replaced with a weighted sum of $C_2$ calculated using separate spin-dependent scattering parameters.  
That calculation would use $\rho_S$, the fraction of pairs in each total spin state S, as the weight.  
For statistical reasons, this analysis will eschew spin-dependent measurements of the $\Lambda\bar{\Lambda}$ scattering parameters, and instead attempt to fit for a spin-averaged value.


\subsubsection{Observables}
\label{sec:Observables}
For pion, kaon, and proton femtoscopic analyses, the scattering lengths and effective range of interaction are well specified by years of scattering data, and they can be fixed to the known values.  
In the case of lambda-(anti-)lambda femtoscopy, the interaction is heretofore largely unstudied.  
For this analysis, the scattering lengths and effective range are left as free parameters.  
Another free parameter is the $\lambda$ parameter, which is a rough measure of the pair purity.  
For example, $\lambda_{\Lambda\Lambda}$ would correspond to the ratio of number of true primary $\Lambda\Lambda$ pairs to the total number of pairs in the correlation function. 
The remaining fraction is comprised of all the other combinations of pair types, such as pairs where one $\Lambda$ is real and primary and the other is fake, or pairs between primary and secondary lambdas.  
The implementation of $\lambda$ parameters is described in Section \ref{sec:TransformedResiduals}, and estimates of the parameters are given in Section \ref{sec:LambdaParams}
The last fit parameter is $R_{\rm inv}$, which characterizes the one dimensional size of the emitting source.  


