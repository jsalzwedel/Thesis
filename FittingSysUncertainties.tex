\subsection{Systematic uncertainties of fits}
\label{sec:SysErrorsFits}

We have two types of systematic uncertainties that play into our fit results: uncertainties coming from our choice of topological and pair cuts and uncertainties related to our fit procedure.
In the following sections, we'll discuss how we quantify those uncertainties.


\subsubsection{Systematic uncertainties from topological and pair cuts}
\label{sec:SysErrorsFitsCuts}

As discussed in Section \ref{sec:SysUncertaintyCF}, there are systematic uncertainties in our results arising from ambiguities in our choice of topological reconstruction cuts and pair cuts.
We analyzed how varying those cuts varied the correlation functions, which resulted in systematic error bars on the correlation functions.
We do not directly include the correlation function systematic error bars in our fits.
Instead, we fit the correlation functions for each cut configuration and then see how the fit results vary.

As discussed earlier, for each type of cut we construct correlation functions using the nominal cut value as well as using $\pm10\%$ of the cut value.
We then fit each of these correlation functions and record the measured fit parameters.
For each type of cut and each parameter, we find the standard deviation of the three measurements.
Then we calculate the rms value of those standard deviations, and quote that as the systematic error.
The results can be found in Table \ref{tab:FitParams}.



\subsubsection{Systematic uncertainties from fit method}
\label{sec:SysErrorsFitsMethod}

%Checking different methods
% Exclude bad methods
% Vary fit range
% Vary other stuff?
% How to combine


We have examined various aspects of the fit process for sources of systematic uncertainty.
We varied the fit range, changed the order of the polynomial background fit, and explored different options of how to do joint fitting of multiple correlation functions.


We explored variations of the upper bound of the fit range: 0-300 MeV, 0-400 MeV, 0-500 MeV.

For the background function $f_{\mathrm{back}}(k^*)$, we tried using a linear background, a quadratic background, and no background ($f_\mathrm{back} = 1$.


% Table? Or put table in next section with results

% Combine with Fit Results section?