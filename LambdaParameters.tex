\subsection{$\lambda$ parameters}
\label{sec:LambdaParams}

We described how $\lambda$ parameters are used in Section \ref{sec:TransformedResiduals}. In this section, we'll estimate the $\lambda$ parameter for the primary-primary correlation function, $\lambda_{\Lambda\Lambda}$, as well as for all the relevant residual correlations.
As stated earlier, $\lambda_\mathrm{XY}$ loosely describes the fraction of pairs in the correlation function composed of particles from species X and species Y.
In this analysis, there are too many contributions from different types of pairs to leave the various $\lambda$ as free parameters.
Instead, we will estimate them beforehand and use those estimates as fixed inputs in our fit procedure.

First, we will develop a model for calculating $\lambda$ parameters. 
In order to do these $\lambda$ calculations, we will need estimates of typical Pb---Pb event yields for $\Lambda$ and $\Lambda$ parent species.
We also need to know how detector reconstruction efficiencies affect the reconstruction rate of $\Lambda$ of different origins.

\subsubsection{Model of $\lambda$ parameters}
\label{sec:LambdaModel}

The experimental correlation function is defined as 
\begin{equation}
\label{eq:ExpCorrelationFunction}
C_{\mathrm{Exp}}(k^*) \equiv \frac{1}{\gamma} \frac{A(k^*)}{B(k^*)},
\end{equation}
where $A(k^*)$ is the same-event relative momentum distribution, $B(k^*)$ is the mixed-event relative momentum distribution, and $\gamma$ is a normalization factor defined to set the high-$k^*$ background region to unity. 
In what follows, we will investigate the relationship between $A(k^*)$ and $B(k^*)$. 


Let us define $N_{ij}$ as the number of $\Lambda\Lambda$ pairs that originated as type $i$ (e.g. $\Lambda\Sigma^0$ or $\Xi^-\Xi^0$) in event $j$.
Next we define $D_{ij}(k^*)$ as the normalized phase-space distribution $\frac{\mathrm{d}N}{\mathrm{d}k^*}$ of pair type $i$ in event $j$. 
Putting these together, we can write the total relative momentum distribution for a single event as
$$A_j(k^*) = \sum_i^m N_{ij} D_{ij}(k^*),$$
where $m$ is the number of different pair types.
Then the total relative momentum distribution, composed of many events, can be written as
$$A(k^*) = \sum_j^n \sum_i^m N_{ij} D_{ij}(k^*),$$
where $n$ is the total number of events.
Similarly, the total relative momentum distribution for mixed-event pairs can be written as
$$B(k^*) = \sum_j^{n'} \sum_i^m N_{ij}' E_{ij}(k^*),$$
where $N_{ij}'$ is the number of pairs of type $i$ in mixed-event $j$, $n'$ is the total number of mixed-events, and $E_{ij}(k^*)$ is the normalized, underlying relative momentum distribution.

Synthesizing Eqns.\ \ref{eq:GeneralizedLambda} and \ref{eq:ExpCorrelationFunction} yields
\begin{equation}
C_{\mathrm{Exp}} = \frac{1}{\gamma} \frac{A(k^*)}{B(k^*)} = \frac{1}{\gamma}\frac{\sum_j^n \sum_i^m N_{ij} D_{ij}(k^*)}{\sum_j^{n'} \sum_i^m N_{ij}' E_{ij}(k^*)} =  \sum_{i}^{m} \lambda_{i} C_{i}(k^*).
\end{equation}
We can look at the component for each type of pair separately:
\begin{equation}
\lambda_i C_i(k^*) = \frac{1}{\gamma}\frac{\sum_j^n N_{ij} D_{ij}(k^*)}{\sum_j^{n'} \sum_i^m N_{ij}' E_{ij}(k^*)}.
\end{equation}

We now make several assumptions. First, it is commonly assumed that the same-event relative momentum distribution is factorizable into a term describing the femtoscopic physics effects and a term describing the simple phase-space of relative momentum (i.e.\ the convolution of two single-particle momentum phase spaces). 
In that case, we can write
\begin{equation}
\label{eq:DEqualCE}
D_i(k^*) = C_i(k^*)E_i(k^*),
\end{equation}
where $C_i(k^*)$ is the correlation function one would measure for pair type $i$ if one could detect said pairs with perfect purity.
Second, we assume that the same-event phase space distribution of a pair type $D_{ij}(k^*)$ is constant across all events that have the same centrality and z-vertex binning. Then we can pull $D_{ij}$ out of the summation, dropping the event index $j$:
\begin{equation}
\lambda_i C_i(k^*) = \frac{1}{\gamma}\frac{D_{i}(k^*)\sum_j^n N_{ij} }{\sum_j^{n'} \sum_i^m N_{ij}' E_{ij}(k^*)}.
\end{equation}
Third, we assume that the mixed-event pair phase space $E_{ij}(k^*)$, which only describes the combinatorics of pair relative momentum, is constant across all pair types and all events. Then we can pull out $E_{ij}$ as well:
\begin{equation}
\lambda_i C_i(k^*) = \frac{1}{\gamma}\frac{D_{i}(k^*)\sum_j^n N_{ij} }{E_{i}(k^*)\sum_j^{n'} \sum_i^m N_{ij}'}.
\end{equation}
Next, we divide out $C_i(k^*) = D_i(k^*) /E_i(k^*)$ on both sides of the equation to get an expression for $\lambda_i$
\begin{equation}
\label{eq:LambdaPartialSolution}
\lambda_i= \frac{1}{\gamma}\frac{\sum_j^n N_{ij} }{\sum_j^{n'} \sum_i^m N_{ij}'}.
\end{equation}
Subject to a choice of normalization parameter $\gamma$, this is an expression for $\lambda$ framed entirely in terms of numbers of pairs of particles in events.  Such an expression can be evaluated using representative Monte Carlo events.

We can clean this expression up further.  It is common practice to use integrated pair counts for correlation function normalization. Often, the numerators and denominators are normalized separately:
\begin{equation}
C(k^*) = \frac{A(k^*)/\sum_j^n \sum_i^m N_{ij}}{B(k^*)/\sum_j^{n'} \sum_i^m N_{ij}'}
\end{equation}
This leads to the normalization parameter
\begin{equation}
\gamma = \frac{\sum_j^n \sum_i^m N_{ij}}{\sum_j^{n'} \sum_i^m N_{ij}'}.
\end{equation}
Substituting this into Eqn.\ \ref{eq:LambdaPartialSolution} yields 
\begin{equation}
\lambda_i = \frac{\sum_j^{n'} \sum_i^m N_{ij}'} {\sum_j^n \sum_i^m N_{ij}} \frac{\sum_j^n N_{ij} }{\sum_j^{n'} \sum_i^m N_{ij}'} = \frac{\sum_j^n N_{ij} }{\sum_j^n \sum_i^m N_{ij}}
\end{equation}
We see that the number of mixed-event pairs drops out of the equation entirely. We can introduce a $1/n$ factor to both the numerator and denominator to get
\begin{equation}
\lambda_i = \frac{\frac{1}{n}\sum_j^n N_{ij} }{\frac{1}{n}\sum_j^n \sum_i^m N_{ij}}.
\end{equation}
The expression $\frac{1}{n} \sum_j^n$ is, of course, an event averaging. We finally conclude that
\begin{equation}
\label{eq:LambdaParSolution}
\lambda_i = \frac{\langle N_{i}\rangle_{\mathrm{event}}} {\langle\sum_i^m N_{i}\rangle_{\mathrm{event}}}.
\end{equation}
In plain terms, the $\lambda$ parameter for pair type $i$ can be computed as the ratio of the average number of pairs of that type per event to the average number of total pairs per event. 
The following sections will discuss how we use Monte Carlo data to estimate these numbers.

\subsubsection{Thermal yield estimates}
\label{sec:ThermalYields}

% Plots

In order to compute the average numbers of pairs per event, as discussed in the previous section, we need two things: a Monte Carlo generator with reliable particle yields and many events. 
HIJING is notoriously unreliable for producing accurate hyperon spectra.  
HIJING Pb---Pb events are generated from a superposition of PYTHIA p---p collisions, so they lack the strangeness saturation of a fully thermalized medium.  
Instead of HIJING, we'll use THERMINATOR 2 (THERMal heavy IoN generATOR version 2)  \cite{Chojnacki:2011hb} to generate events.

THERMINATOR 2 is a Monte Carlo event generator designed to use statistical hadronization to produce hadronic freeze-out spectra of relativistic heavy-ion collisions. 
The user selects a hypersurface model and freeze-out model (several of which are included in the package), as well as rapidity range and the type of primordial particle multiplicity distribution (e.g.\ Poissonian). 
One of the crucial features of THERMINATOR is the inclusion of a large number of resonances from the particle data tables. 
These play an important role in the analysis, as a sizable number of these resonances decay into hyperons, thus boosting the hyperon yields. 
The resonances that undergo strong-decay into hyperons do so on time-scales on the order of $\mathrm{fm}/c$ or tens of $\mathrm{fm}/c$. 
These decays happen on the same time-scale as the hadronic rescattering phase, so it is possible that the particle has decayed into its hyperon daughter before kinetic freeze-out. 
In that case, it is the final-state interaction of the hyperon daughter that is relevant to femtoscopic analyses, not the interactions of the resonant parent.  
In this analysis, we will assume that all short-lived resonances decay before kinetic freezeout. 
Under that assumption, we can treat the daughters of these resonances as primary particles. 

In this analysis, we used the Lhyquid3v freeze-out model for 2.76 TeV Pb--Pb events. This freeze-out model is not packaged with THERMINATOR 2 --- it was obtained via correspondance with the group that authored the THERMINATOR 2 package.
There are several hypersurfaces available for this model.
We chose the one with the most-central impact parameter, $b = 2.3$ fm, and a hadronic freeze-out temperature of 140 MeV.
While the default is to use a rapidity window of $\pm$ 12 units, we set our rapidity window to $\pm$ 2 units to conserve disk space.

Using THERMINATOR, we generated 2000 events. 
In those events, we found all the primary and secondary $\Lambda$, ignoring those that fall outside the pseudorapidity and $p_\mathrm{T}$ cuts used in the data collection.
For each $\Lambda$ we found in a THERMINATOR event, we generated a random number between 0 and 1. 
We ignored the $\Lambda$ if the generated number was higher than the $\Lambda$'s estimated reconstruction efficiency of $\sim15\%$ (see Section \ref{sec:ReconstructionEff} for more details on efficiency estimates). 
We tallied the reconstructed $\Lambda$ and noted their origins. 
We also added fake $\Lambda$ counts to each event to account for reconstruction purity.
As discussed in Section \ref{sec:Recon}, we have an approximately 95\% purity.
To add fakes, we looked at the average number of real particles tallied per event and determined the average number of fakes corresponding to a 5\% impurity.
Then for each event, we generated a number of fakes according to a Poissonian distrubution peaked at the estimated value.
The results without efficiency corrections are visible in Figure \ref{fig:LambdaYieldsNoEfficiency}. 
The results with efficiency corrections can be seen in Figure \ref{fig:LambdaYieldsWithEfficiency}.


\begin{figure}[hbtp]
\includegraphics[width=36pc]{Figures/YieldAndEff/2016-08-06-LambdaYieldsNoEfficiency.pdf}
\caption[$\Lambda$ yields from THERMINATOR 2 without efficiency corrections]{
Expected yield of $\Lambda$ in each event arising from different sources, as predicted by the thermal hadronization model in THERMINATOR 2 \cite{Chojnacki:2011hb}. The x-axis shows the different origins: primary $\Lambda$, secondary $\Lambda$ coming from various hyperons, and fakes (misidentified particles).  $\Lambda$ coming from the strong decay of short-lived resonances are included in the primary $\Lambda$ count. The number of fakes in each event are generated via a Poissonian distribution, assuming an $\approx 5$\% impurity. Reconstruction inefficiencies, which naturally reduce the number of reconstructed particles, are not included in this plot.
}
\label{fig:LambdaYieldsNoEfficiency}
\end{figure}


\begin{figure}[hbtp]
\includegraphics[width=36pc]{Figures/YieldAndEff/2016-08-06-LambdaYieldsWithEfficiency.pdf}
\caption[$\Lambda$ yields from THERMINATOR 2 with efficiency corrections]{
Expected yield of $\Lambda$ in each event arising from different sources, as predicted by the thermal hadronization model in THERMINATOR 2 \cite{Chojnacki:2011hb}. The x-axis shows the different origins: primary $\Lambda$, secondary $\Lambda$ coming from various hyperons, and fakes (misidentified particles).  $\Lambda$ coming from the strong decay of short-lived resonances are included in the primary $\Lambda$ count. The number of fakes in each event are generated via a Poissonian distribution, assuming an $\approx 5$\% impurity. Reconstruction efficiency corrections (estimated in Section \ref{sec:ReconstructionEff}) are included in these yields.
}
\label{fig:LambdaYieldsWithEfficiency}
\end{figure}




Finally we used the tallied $\Lambda$ to compute the number of pairs of each type per event (including real-real, real-fake, and fake-fake), along with the total number of pairs in each event, and we recorded that information.
Events with no pairs (i.e. events that had one or fewer reconstructed $\Lambda$) were not included in event averaging, as they are skipped in the correlation function measurement of the real data.

% Plot or table of avg yields, fakes, etc. Before and after efficiencies.


The following section describes how we estimated the reconstruction efficiencies that were used in this process.
The results of the $\lambda$ parameter calculations can be found in \ref{sec:LambdaParamEstimates}.

\subsubsection{Reconstruction efficiency effects on relative particle yields}
\label{sec:ReconstructionEff}

  

In the previous section, we described how we used the THERMINATOR 2 event generator to estimate yields of primary and secondary $\Lambda$.
Those yields were produced according to a statistical hadronization model.
However, in real data we do not reconstruct perfect thermal spectra of particles. 
Instead, the yields are tempered by reconstruction efficiencies.
The ALICE detector does not have perfect track resolution, so some tracks are misidentified or lost.
We also employ topological reconstruction cuts to remove fake $\Lambda$ candidates.
Because of the finite detector resolution, some of the real $\Lambda$ are removed as well.
In addition, the $\Lambda$ topological reconstruction cuts have been selected to preferentially remove secondary $\Lambda$ coming from $\Xi$ and $\Omega$ decays (see Section \ref{sec:Recon}).
This is not 100\% effective at removing secondary $\Lambda$, and it also causes some primary $\Lambda$ to be thrown out.
Finally, the branching ratio of the $\Lambda$ weak decay results in roughly a third of the $\Lambda$ decaying into untrackable neutral particles.
These effects naturally result in smaller measured yields and thus smaller quantities of pairs.
The reconstruction efficiency is an important component of our estimation of $\lambda$ parameters.
To account for the effects of reconstruction efficiency on the $\lambda$ parameters, we artificially introduced reconstruction efficiencies into the THERMINATOR data. 
In this section, we discuss how we used Monte Carlo HIJING events with simulated detector reconstruction to estimate these reconstruction efficiencies.

To investigate the influence of reconstruction efficiencies on particle yields, we looked at the reconstruction rates of the different particle types using the HIJING run specified in Section \ref{sec:DataSelection}.  
The MC-truth yields of the V0s were examined at three stages of reconstruction:
\begin{enumerate}
\item In the raw MC event before any detector simulation was added to the MC particles.  
Here the daughters of the V0s are required to be in mid-rapidity and surpass a minimum $p_\mathrm{T}$ value, so as to only look at V0s that could reasonably be found.
\item In the V0 finder before any analysis-side reconstruction cuts were employed.
\item After topological reconstruction cuts were made.
\end{enumerate}
At each stage, the MC truths of all remaining particles of the following types were counted: primary $\Lambda$; secondary $\Lambda$ from $\Sigma^0$, $\Sigma^*$, $\Xi^0$, $\Xi^-$, $\Omega$, and other sources; the respective anti-particles; $\mathrm{K}^0_{\mathrm{S}}$ misidentified as $\Lambda$ or $\bar{\Lambda}$; other misidentified V0s; and fake V0s reconstructed in the V0 finder.  
The resulting yields are visible in Figure \ref{fig:MCYields}.

\begin{figure}[hbtp]
\includegraphics[width=36pc]{Figures/2014-02-03-MCYields.pdf}
\caption[$\Lambda$ MC Yields at Different Reconstruction Stages]{
MC truth yields of $\Lambda$ at different stages of reconstruction.
The x-axis shows the parentage of the reconstructed particle. 
The $\Lambda$ and $\bar{\Lambda}$ bins are primary particle yields.  
Other bins show the number of $\Lambda$ ($\bar{\Lambda}$) that come from decays of other particles. 
The $\mathrm{K^0_s}$ bin shows the number of kaons that were misidentified as $\Lambda$ and $\bar{\Lambda}$.
The black points show the yields that exist in the underlying event, the red points show the true parentage of V0-candidates that have been found by the offline V0 finder, and the blue points show yields after all topological reconstruction cuts have been imposed.
Only midrapidity particles that exceed a minimum $p_\mathrm{T}$ cut have been examined, and injected signals have been excluded from this analysis.}
\label{fig:MCYields}
\end{figure}

The results of Figure \ref{fig:MCYields} can be better interpreted by taking the ratios of the yields at different stages.  

Figure \ref{fig:OriginToV0} shows one such ratio.
Here, one can see the fraction of $\Lambda$ of different parentage that were generated in the event that were also recognized by the V0 finder.
Note that roughly a third of them are already lost at this stage because of the $\Lambda \rightarrow \mathrm{n}\pi^0$ decay channel.
% lambda from seconary decay have higher pT and thus are more likely to be found than primary lambda



Figure \ref{fig:V0ToMassCut} shows the fraction of V0 candidates in the V0 finder that survived all the reconstruction cuts.  
From this, one can see how successful the topological cuts are at selecting (anti)$\Lambda$ of various origins.  
Here you can see the selection cuts at work --- primary $\Lambda$, $\Lambda$ from $\Sigma^0$, $\Lambda$ from resonance decays have a much better survival rate than $\Lambda$ from weak decays.
%With different topological cuts made, a before-and-after comparison of this plot shows the efficacy of the new cuts at removing secondary V0s vs primary V0s, as in Figure \ref{V0ToMassCutWithDifferentVarBin}.

The final ratio worth considering is Figure \ref{fig:OriginToMassCut}, which shows the reconstruction efficiency from the beginning (particles in the underlying event) to the end (all particles passing all cuts).
This information, which is just the convolution of the previous two figures, conveys the overall reconstruction efficiency of $\Lambda$ of different origins.
This efficiency encompasses not only the results of the analysis-side cuts, but also the natural tracking efficiencies of ALICE detector.  
One can see that $\Lambda$ of different origins all have approximately the same chance of getting reconstructed.
When applying the efficiency correction to the yields in Section \ref{sec:ThermalYields}, we used a flat 15\% efficiency for $\Lambda$ of all origins.

\subsubsection{A note on injected Monte Carlo signals}
\label{sec:InjectedMCSignals}

It should be noted that this MC run, LHC12a17a\_fix, has injected $\Xi^0$, $\Xi^-$, and $\Omega$ signals, but not their respective antiparticles.  T
he disparity can be seen in Figure \ref{fig:MCEfficiencyWithInjected}, which shows the average reconstruction efficiencies of $\Lambda$ ($\bar{\Lambda}$) broken down in terms of the Monte Carlo truth of each V0's parent species. 
The reconstruction efficiency is defined as the ratio of the number of reconstructed $\Lambda$ ($\bar{\Lambda}$) of each origin type versus the number of V0 candidates of each origin in the V0 finder.  
A clear disparity can be seen between the reconstruction efficiencies of $\Lambda_{\Xi}$ and $\Lambda_{\Omega}$ and their respective antiparticles.  
(This plot is a few months out of date - the reconstruction cuts employed in this figure are somewhat looser than those employed in the current analysis.)

The origin of this disparity has not been investigated in great detail, though it is suspected that the disparity arises from differences between the $p_\mathrm{T}$ spectra of the injected V0s and the underlying HIJING particles.  
Because the efficacy of the various reconstruction cuts and the detector efficiency is $p_\mathrm{T}$ dependant, the average reconstruction efficiency of injected particles may differ from the base HIJING particles.  
Figure \ref{fig:MCEfficiencyWithInjected} therefore shows a disparity between multistrange hyperon and antihyperon efficiency because the hyperons include a mix of injected and underlying particles, while the antihyperons come only from the underlying event. 
The analysis was performed again with injected signals removed (see Figure \ref{fig:MCEfficiencyNoInjected}).  
Without the injected signals, reconstruction efficiencies for the multi-strange particles and anti-particles are seen to be in line.  
Subsequent studies of reconstruction cuts and efficiency were performed without injected signals such that secondary $\Lambda$ and $\bar{\Lambda}$ efficiencies would be consistent.

\begin{figure}[hbtp]
\includegraphics[width=36pc]{Figures/2014-04-20-Efficiency-WithInjectedSignals-OldCuts_2013-09-04Run.pdf}
\caption[$Lambda$ reconstruction efficiencies with injected signals]{Average reconstruction efficiencies of $\Lambda$ ($\bar{\Lambda}$) broken down in terms of the Monte Carlo truth of each V0's parent species.  
Reconstruction efficiency is defined as the ratio of the number of reconstructed $\Lambda$ ($\bar{\Lambda}$) of each origin type versus the number of V0 candidates of each origin in the V0 finder.  
The analysis was performed for HIJING events with injected hyperon signals.  
The events included injected $\Lambda$, $\bar{\Lambda}$, $\Xi$, and $\Omega$, but no $\bar{\Xi}$ or $\bar{\Omega}$.  
A clear disparity can be seen between the reconstruction efficiencies of $\Lambda_{\Xi}$ and $\Lambda_{\Omega}$ and their respective antiparticles.  
This plot is a few months out of date - the reconstruction cuts employed in this figure are somewhat looser than those employed in the current analysis.}
\label{fig:MCEfficiencyWithInjected}
\end{figure}

\begin{figure}[hbtp]
\includegraphics[width=36pc]{Figures/2014-04-20-Efficiency-NoInjectedSignals-OldCuts_2013-11-19Run.pdf}
\caption[$Lambda$ reconstruction efficiencies without injected signals]{Average reconstruction efficiencies of $\Lambda$ ($\bar{\Lambda}$) broken down in terms of the Monte Carlo truth of each V0's parent species.  
Reconstruction efficiency is defined as the ratio of the number of reconstructed $\Lambda$ ($\bar{\Lambda}$) of each origin type versus the number of V0 candidates of each origin in the V0 finder.  
The analysis was performed for HIJING events with injected hyperon signals.  
However, injected hyperon signals have been excluded from this plot.  
The various secondary $\Lambda$ are seen to have approximately the same reconstruction efficiency as respective antiparticles.  
This plot is a few months out of date - the reconstruction cuts employed in this figure are somewhat looser than those employed in the current analysis.}
\label{fig:MCEfficiencyNoInjected}
\end{figure}




\subsubsection{Estimates of \texorpdfstring{$\lambda$}{lambda} parameters}
\label{sec:LambdaParamEstimates}

In this section, we estimate of the relevant $\lambda$ parameters involved in fitting these systems.  
$\lambda$ can roughly be understood to be the pair purity of the system that is being fit, i.e.\ the fraction of true pairs over total pairs.  
There are additional factors that can affect $\lambda$ such as source coherence or non-gaussianity of underlying source, but these complications will be neglected in this analysis.

Many femtoscopic analyses utilize a single $\lambda$ value in their fit procedure as an indication of their overall pair purity.  
This is fine in the cases where the impure pairs (i.e.\ the other $1 - \lambda$ fraction of pairs) are uncorrelated.
But if there are residual correlations, then it becomes necessary estimate their contributions to the correlation function.
Additional $\lambda$ parameters can be attributed to each residual correlation according to the fraction of total pairs of that type.
In principle, all these $\lambda$ can be left as free fit parameters.
In practice, that results in fits that are far too unconstrained to converge properly.
In this analysis, we will estimate the $\lambda$ parameters for each relevant residual correlation, and use those numbers as fixed parameters in the fit.

%A more refined estimate could be made by convoluting ALICE particle spectra with analysis-specific reconstruction efficiencies to determine analysis-specific yields of the various particle types.  
%That avenue will be discussed more in Section \ref{sec:ReconstructionEff}. 

To understand the mathematical motivation behind the $\lambda$ parameters, let us consider the simple case where the signal pairs of the correlation function come from a single event.  
Within this event, all combinations of reconstructed V0s are paired together and binned according to their $k^*$ values.  
For this simple example, let's assume that the reconstructed V0s consist of primary $\Lambda$, secondary $\Lambda$, and fake $\Lambda$.  
Let us assume that origin of the different $\Lambda$ cannot be determined - the analysis only knows that a given V0 appears to be a $\Lambda$.  
As mentioned above, the total correlation function will be a linear combination of the component correlation functions (primary-primary, primary-secondary, primary-fake, fake-fake, etc.).  
The $\lambda$ parameters for each component correlation are given by the fraction of total pairs of that type.  
For example, if there are ten distinct $\Lambda\Sigma$ pairs and one hundred total pairs, then $\lambda_{\Lambda\Sigma} = 1/10$.  
If all pair types are accounted for, 
\begin{equation}
\label{eq:SumLambdaUnity}
\sum\limits_{ij}\lambda_{ij} = 1.
\end{equation}

We can use the following equations to estimate the $\lambda$ parameters for different types of pairs. 
These equations take the general form 
\begin{equation}
\lambda = \frac{\mathrm{Specific Pairs}}{\mathrm{Total Pairs}},
\end{equation}
where the calculations for numbers of pairs are simple combinatorics.  If more than one event is being used (as is the case in the data), the equation becomes
\begin{equation}
\lambda = \frac{\langle\mathrm{Specific Pairs}\rangle}{\langle\mathrm{Total Pairs}\rangle},
\end{equation}
where the average is done over events.

The $\lambda$ parameter for two identical particles (e.g. both primary $\Lambda$) is given by
\begin{equation}
\label{eq:LambdaIdentical}
\lambda_{ii} = \frac{N_i}{T}\frac{(N_i -1)/2}{(T-1)/2} = \frac{N_i}{T}\frac{(N_i -1)}{(T-1)}
\end{equation}
where $N_i$ is the number of particles of species $i$, $T = \sum\limits_i N_i$ is the total number of particles, and the factors of $1/2$ remove double counting.  
For two particles with different origins (e.g. a primary $\Lambda$ and a $\Lambda$ from a $\Sigma^0$ decay)
\begin{equation}
\lambda_{ij} = \frac{N_i}{T} \frac{N_j}{(T-1)/2}
\end{equation}
for $i \neq j$.  
$\lambda$ can also be calculated for pairs that are distinguishable, such as primary $\Lambda$ and primary $\bar{\Lambda}$:
\begin{equation}
\lambda_{i\bar{j}} = \frac{N_i N_{\bar{j}}}{T\bar{T}}
\end{equation}
where $T$ is the sum of all indistinguishable particles and $\bar{T}$ is the sum of all indistinguishable antiparticles.



%One should note here that the above $\lambda$ equations technically apply to some sort of single-event correlation function.  
%The $\lambda$ parameters measured or estimated for a million-event correlation function would roughly correspond to an event-averaged value.  
%Nonetheless, we can attempt to employ these formulae to obtain ballpark estimates of the $\lambda$ parameters, given estimates of the yields of each type of particle.
%
%As the goal of this analysis is to measure the correlations of primary $\Lambda$, it is important to know the relative yields of secondary $\Lambda$.  
%We can obtain an estimate \cite{Florkowski:2010zz} of the yields of different types of particles at mid-rapidity from a thermal model using
%\begin{equation}
%\frac{1}{m_{\mathrm{T}}}\frac{dN}{dm_{\mathrm{T}}} \propto \exp{(-m_\mathrm{T}/T)}
%\end{equation}
%where the transverse mass $m_\mathrm{T} = \sqrt{m^2_{\mathrm{inv}} + p^2_{\mathrm{T}}}$ and $T$ is the chemical-feezeout temperature.  
%Integrating over $m_\mathrm{T}$ gives $N \propto (m_\mathrm{inv} + T) \exp{(-m_\mathrm{inv}/T)}$.  
%Therefore, the yield of any particle species $i$ relative to the $\Lambda$ yield is
%\begin{equation}
%\frac{N_i}{N_\Lambda} \approx \frac{m_i + T}{m_\Lambda + T} \exp{((m_\Lambda - m_i)/T)}.
%\end{equation}
%
%Particles that decay into $\Lambda$ include $\Sigma^0$, $\Xi^0$, $\Xi^-$, and $\Omega^-$.  
%Assuming a freezeout temperature of 165 MeV, and taking into account that the $\Omega$ has a branching ratio of ~68\% and the others ~100\%, we can then estimate that for every 100 $\Lambda$, there will be approximately 67 $\Sigma$, 34 of each type of $\Xi$, and 3 $\Omega$.
%
%Before calculating the $\lambda$ parameters, it is also necessary to know approximately how many fake $\Lambda$ there are.  
%In Section \ref{sec:Recon}, we estimated that our signal quality $P$ was $P = real/(real + background) \approx 0.82$ (Note: calculations were performed using a now outdated purity estimate).  
%Taking $real$ to be the sum of all the primary and secondary $\Lambda$, we calculate that $\frac{background}{real} = P^{-1}-1 \approx 0.22$.  
%With the above yields totalling 238 $\Lambda$, we would expect to see an additional 52 fake $\Lambda$.  
%Based on these values, we can now estimate the $\lambda$ parameters for all pair types.
%
%\begin{center}
%\begin{tabular}{|l|l|c|}
%\hline
%					& 	$\lambda$	&	Cumulative total \\ \hline
%$\Lambda\Lambda$   	&	0.12			&	0.12 \\ \hline
%$\Lambda\Sigma^0$  	&	0.16			&	0.28 \\ \hline
%$\Lambda\Xi^0$     	&	0.08			&	0.36 \\ \hline
%$\Lambda\Xi^-$     	&	0.08			&	0.44 \\ \hline
%$\Lambda\Omega$    	&	0.007		&	0.45 \\ \hline
%$\Sigma^0\Sigma^0$ 	&	0.05			&	0.50 \\ \hline
%$\Sigma^0\Xi^0$    	&	0.05			&	0.55 \\ \hline
%$\Sigma^0\Xi^-$    	&	0.05			&	0.61 \\ \hline
%$\Sigma^0\Omega$   	&	0.004		&	0.61 \\ \hline
%$\Xi^0\Xi^0$       	&	0.01			&	0.63 \\ \hline
%$\Xi^0\Xi^-$ 		&	0.03			&	0.65 \\ \hline
%$\Xi^0\Omega$ 		&	0.002		&	0.66 \\ \hline
%$\Xi^-\Xi^-$ 		&	0.01			&	0.67 \\ \hline
%$\Xi^-\Omega$ 		&	0.002		&	0.67 \\ \hline
%$\Omega\Omega$ 		&	0.00			&	0.67 \\ \hline
%\end{tabular}
%\end{center}
%
%These estimates have several interesting characteristics.  
%First, the sum of the primary correlation all the residual correlations totals to 0.67.  
%The other 33\% is lost to the background.  
%Note that feedup from residual p$\Lambda$ correlations would be included in this 33\%, though feedup has otherwise not yet been explored in this analysis. 
%Another striking feature of these data is that the contribution from $\Lambda\Sigma$ is actually larger than the contribution from $\Lambda\Lambda$.  
%In other words, if the hyperon yields and reconstruction efficiencies of the true PbPb data are reminiscent of the yields above, then this analysis is measuring the $\Lambda\Sigma$ correlation more than the $\Lambda\Lambda$ correlation.  
%It should also be noted that contributions from $\Lambda\Xi$, $\Sigma\Sigma$, and $\Sigma\Xi$ are appreciable.
%
%For the the $\Lambda\bar{\Lambda}$ analysis, estimates of the $\lambda$ parameters work out to be approximately the same as in the table above.
