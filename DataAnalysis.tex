\section{Data analysis}




%In this section, we will discuss the methodology used to make momentum resolution corrections. 
%
%Traditionally, the momentum resolution correction is a multiplicative correction factor in $k^*$ that is used to modify the experimental correlation function:
%\begin{equation}
%C_\mathrm{corrected}(k^*) = C_{\mathrm{experimental}}(k^*)*F_\mathrm{correction}(k^*).
%\end{equation}
%Here, the correction factor is constructed as a ratio of the true correlation function divided by the reconstructed correlation function (both taken from MC)
%\begin{equation}
%F_\mathrm{correction}(k^*)=\frac{C_\mathrm{true}(k^*)}{C_\mathrm{recon}(k^*)}.
%\end{equation}
%The correlation functions are constructed via
%\begin{equation}
%\label{eq:Ctrue}
%C_{\mathrm{true}} = \frac{ N(k^*_{\mathrm{true}},W(C_{\mathrm{th}}(k^*_{\mathrm{true}})))}{N(k^*_{\mathrm{true}},1)}
%\end{equation}
%\begin{equation}
%\label{eq:Crecon}
%C_{\mathrm{recon}} = \frac{ N(k^*_{\mathrm{recon}},W(C_{\mathrm{th}}(k^*_{\mathrm{true}})))}{N(k^*_{\mathrm{recon}},1)},
%\end{equation}
%where $W(C_\mathrm{th}(k^*_\mathrm{true}))$ is the weight of the theoretically predicted correlation function for a given true $k^*$ bin.
%
%The usual strategy is iterative: fit the uncorrected data to get a rough estimate of the theoretical correlation function, feed that fit into the MC analysis as the "true" correlation function weight, construct the correction factor and use it to correct the data, fit the corrected data, and repeat until the results converge.
%One complication with this method is simply that the iterative process is slow --- it requires many runs over the data with different weights.
%
%
%To obtain the width of the momentum smearing, a MC analysis is performed in which $k*_{\mathrm{smeared}} = k^*_{\mathrm{reconstructed}} - k^*_{\mathrm{truth}}$ is binned.  $k^*_{\mathrm{truth}}$ is determined from looking at the associated AliAODMCParticles of the reconstructed tracks.  
%The analysis is only performed for real V0s.  
%Fake V0s (as defined in Section \ref{sec:Recon}) were omitted, as they have no true momenta.  
%The analysis did not distinguish between primary and secondary $\Lambda$, as both are affected by momentum resolution effects and both appear in the data.  
%The binning was done performed only for mixed-event pairs, as those pairs will reflect the momentum resolution effects on the overall phase space without including any complications from physics interactions.  
%At this time, the binning was done in a centrality integrated fashion -- future studies could look to see if and how the effects differ between centralities.  
%After obtaining this distribution for each pair type ($\Lambda\Lambda$, $\bar{\Lambda}\bar{\Lambda}$, and $\Lambda\bar{\Lambda}$), the each distribution was fit with a Gaussian in order to extract a characteristic width of the momentum smearing.  
%One such fit can be seen in Figure \ref{fig:MomSmearingFit}.  
%The large $\chi^2 /\mathrm{ndf}$ indicates that the smearing isn't well described by a Gaussian.  
%Part of that may be due to the $p_\mathrm{T}$ and centrality integrated nature of the histogram.  
%Nonetheless, a Gaussian width has been extracted that roughly characterizes the width of the smearing.  
%Each pair type returned a width that was within a couple percent of $10\ \mathrm{MeV}/c$.
%
%As stated above, when constructing $C_\mathrm{true}$ and $C_\mathrm{recon}$, we used an analytic form of the correlation functions.  
%While the form is well specified (see Section \ref{sec:AnalyticModel}), it was necessary to assume values for the various parameters of the equation: radius, pair fraction ($\lambda$), scattering lengths, effective range of interaction.  
%Typically, thorough evaluations of the resolution correct use an iterative procedure as part of the process of fitting the correlation functions.  
%Values are assumed for the parameters, resolution corrections are made, and fits to the data are performed.  
%Then the process repeats.  
%As the current analysis is not yet at the stage of fitting, we've settled on a "best guess" for the values used here: radius $=2.5$ fm; $\lambda = 0.3$; real part of scattering length $= -0.5$ fm; imaginary part of scattering length $= 0$ fm for $\Lambda\Lambda$ and $\bar{\Lambda}\bar{\Lambda}$, and $0.5$ fm for $\Lambda\bar{\Lambda}$; and the effective range $=3$ fm.  
%These numbers were chosen agnostic of the centralities being fit, where very early fit attempts have estimated radii between 2.7 fm and 2.3 fm.  
%
%Figures \ref{fig:MomCorrectionFactorLLAA} and \ref{fig:MomCorrectionFactorLA} show the momentum resolution correction histograms for $\Lambda\Lambda$/$\bar{\Lambda}\bar{\Lambda}$ and $\Lambda\bar{\Lambda}$ respectively.  
%The dip in the correction factor at low $k^*$ shows that the effect of the momentum resolution on the reconstructed correlation function is to flatten (i.e. raise towards unity) the correlation function in that region).  
%The correction factor serves to undo that flattening.  
%At its largest it is about a three percent correction for $\Lambda\Lambda$/$\bar{\Lambda}\bar{\Lambda}$.  
%The effect is wider but also shallower for $\Lambda\bar{\Lambda}$, where at its max it is less than a one percent correction.  
%In this preliminary momentum resolution correction, the same resolution correction will be applied to each centrality range for a given pair type.  
%
%\begin{figure}
%\includegraphics[width=36pc]{Figures/2014-05-10-PrelimCorrectionFactorLLAA.pdf}
%\caption[Relative momentum correction factor for $\Lambda\Lambda$ and $\bar{\Lambda}\bar{\Lambda}$]{A histogram of the relative momentum correction factors used to correct $\Lambda\Lambda$ and $\bar{\Lambda}\bar{\Lambda}$ correlation functions.  
%The corrected correlation function is found by multiplying each $k^*$ bin of the experimentally measured correlation function by the associated bin of the correction factor.  
%The dip in the correction factor at low $k^*$ shows that the effect of the momentum resolution on the reconstructed correlation function is to flatten (i.e. raise towards unity) the correlation function in that region).  
%The correction factor serves to undo that flattening.  
%At most it is about a three percent correction.}
%\label{fig:MomCorrectionFactorLLAA}
%\end{figure}
%
%\begin{figure}
%\includegraphics[width=36pc]{Figures/2014-05-10-PrelimCorrectionFactorLA.pdf}
%\caption[Relative momentum correction factor for $\Lambda\bar{\Lambda}$]{A histogram of the relative momentum correction factors used to correct $\Lambda\bar{\Lambda}$ correlation functions.  
%The corrected correlation function is found by multiplying each $k^*$ bin of the experimentally measured correlation function by the associated bin of the correction factor.  
%Except for the lowest bin, the dip in the correction factor at low $k^*$ shows that the effect of the momentum resolution on the reconstructed correlation function is to flatten (i.e. raise towards unity) the correlation function in that region).  
%The correction factor serves to undo that flattening.  
%The effect is seen to be less than a one percent correction}
%\label{fig:MomCorrectionFactorLA}
%\end{figure}
%
%Because all the centralities are receiving the same multiplicative (and therefore commutative) correction, the shape of the final, centrality-merged, momentum-corrected correlation functions will be the same whether the correction is applied to each 5\% centrality bin or to the merged correlation functions.  
%For the purpose of determining systematic uncertainties, the correction will be done on the merged correlation functions.  
%The size and shape of the momentum smearing is highly dependent on the analytic form of the correlation function, and it is therefore dependent on the chosen values of the parameters.  
%This sensitivity, along with the currently centrality-independent nature of the correction, will be accounted for in the systematic uncertainties by applying a conservative (i.e. large) uncertainty for this effect.  
