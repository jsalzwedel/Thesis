\subsection{General Details}
\label{sec:GeneralDetails}

\subsubsection{Data selection}
\label{sec:DataSelection}
The analysis was performed on 31 July, 2013 using AliRoot v5-04-64-AN. The data were from PbPb collisions at $\sqrt{s_{NN}}=2.76$  TeV taken from the LHC11h (AOD115) pass 2 reconstruction. 
Runs were selected if they were listed with a pass 2 global quality of 1 ("Good run") in the Run Condition Table. 
Approximately 66 million combined central, semi-central and minimum bias events were analyzed. 
The following runs were used, which comprise both the positive (++) and negative magnetic (--) field orientations:

170593, 170572, 170388, 170387, 170315, 170313, 170312, 170311, 170309, 170308, 170306, 170270, 170269, 170268, 170230, 170228, 170207, 170204, 170203, 170193, 170163, 170159, 170155, 170091, 170089, 170088, 170085, 170084, 170083, 170081, 170040, 170027, 169965, 169923, 169859, 169858, 169855, 169846, 169838, 169837, 169835, 169591, 169590, 169588, 169587, 169586, 169557, 169555, 169554, 169553, 169550, 169515, 169512, 169506, 169504, 169498, 169475, 169420, 169419, 169418, 169417, 169415, 169411, 169238, 169167, 169160, 169156, 169148, 169145, 169144, 169138, 169099, 169094, 169091, 169045, 169044, 169040, 169035, 168992, 168988, 168826, 168777, 168514, 168512, 168511, 168467, 168464, 168460, 168458, 168362, 168361, 168342, 168341, 168325, 168322, 168311, 168310, 168115, 168108, 168107, 168105, 168076, 168069, 167988, 167987, 167985, 167920, 167915

Analysis was also performed on the LHC12a17a\_fix Monte Carlo HIJING run, which corresponds to the ++ field orientation events of the LHC11h anchors.  
Approximately 650 thousand events were analyzed from runs

170593, 170572, 170388, 170387, 170315, 170313, 170312, 170311, 170309, 170308, 170306, 170270, 170269, 170268, 170230, 170228, 170207, 170204, 170203, 170193, 170163, 170159, 170155, 170091, 170089, 170088, 170085, 170084, 170083, 170081, 170040, 170027, 169965, 169923, 169859, 169858, 169855, 169846, 169838, 169837, 169835

In all cases, the z-position of the primary vertex was required to be within 10 cm of the center of the ALICE detector for an event to be selected.  

No centrality flattening is currently employed in the analysis of this data, though it is planned to be completed eventually.  
Centrality flattening is a procedure by which some events are rejected to ensure that all centralities within a given centrality window (e.g. $0-10\%$) are represented with equal frequency.  
Centrality flattening is not expected to have a significant impact on the correlation function results, as it merely remove some statistics to make each 1\% centrality bin more evenly weighted in the resulting data.  
For instance, before flattening, the ALICE Pb---Pb $K^0_\mathrm{s}K^0_\mathrm{s}$ analysis had an average centrality of 4.9\% for the 0-10\% correlation function.  
After flattening, the average became 5\%.  
This should result in essentially no change to extracted radii or scattering lengths.

\subsubsection{Detector use}
Both the data analysis as well as the MC analysis utilized information from several different detectors.  
Event centrality was measured using the V0 detector with a timing cut from the Zero Degree Calorimeter.  
The Inner Tracking System (ITS), Time Projection Chamber (TPC), Transition Radiation Detector (TRD), and Time of Flight detector together provided global tracking for the daughter particles.  
Particle identification (PID) was performed using the TPC, as well as using the Time of Flight detector (TOF) when possible.

\subsubsection{Analysis code}
The analysis task used is called AliAnalysisV0Lam.cxx, with the V0 reconstruction code included in AliAnalysisV0LamCutProcessing.cxx.  
The code for this analyis is not currently committed to AliRoot.