\section{General details}
\label{sec:GeneralDetails}

\subsection{The ALICE detector}
\label{sec:ALICEDetector}


\subsection{Data selection}
\label{sec:DataSelection}

The $\sqrt{s_{NN}}=2.76$ TeV Pb-Pb collision data analyzed in this analysis were recorded at the end of 2011.
This dataset, LHC11h, is comprised of a number of runs.
A "run" refers to the data taken during a continuous recording session of the ALICE detector, as well as to the duration of the recording session.
Runs can include anything from minutes to days of data collection.
Not all runs are used in this analysis.
For example, a particular run might only exist for detector calibration purposes, or a subdetector essential for this analysis might have been offline during data collection.
Runs were selected if they were listed with a Reconstruction Pass 2 global quality of 1 ("Good run") in the ALICE Run Condition Table. 
Approximately 45 million combined central, semi-central and minimum bias events were analyzed. 
The following runs were used:

170593, 170572, 170388, 170387, 170315, 170313, 170312, 170311, 170309, 170308, 170306, 170270, 170269, 170268, 170230, 170228, 170207, 170204, 170203, 170193, 170163, 170159, 170155, 170091, 170089, 170088, 170085, 170084, 170083, 170081, 170040, 170027, 169965, 169923, 169859, 169858, 169855, 169846, 169838, 169837, 169835, 169591, 169590, 169588, 169587, 169586, 169557, 169555, 169554, 169553, 169550, 169515, 169512, 169506, 169504, 169498, 169475, 169420, 169419, 169418, 169417, 169415, 169411, 169238, 169167, 169160, 169156, 169148, 169145, 169144, 169138, 169099, 169094, 169091, 169045, 169044, 169040, 169035, 168992, 168988, 168826, 168777, 168514, 168512, 168511, 168467, 168464, 168460, 168458, 168362, 168361, 168342, 168341, 168325, 168322, 168311, 168310, 168115, 168108, 168107, 168105, 168076, 168069, 167988, 167987, 167985, 167920, 167915

Approximately a third of these runs were taken with the magnetic field of the ALICE detector in the positive configuration, and two-thirds had the magnetic field in the negative configuration.
Having data from both configurations allows for consistency checks; one can see if physics results have any dependence on the detector's magnetic field.

Analysis was also performed on the Monte Carlo HIJING dataset, LHC12a17a\_fix.
The approximately $4.5*10^5$ events in this dataset were reconstructed using calibration information from the following LHC11h runs

170593, 170572, 170388, 170387, 170315, 170313, 170312, 170311, 170309, 170308, 170306, 170270, 170269, 170268, 170230, 170228, 170207, 170204, 170203, 170193, 170163, 170159, 170155, 170091, 170089, 170088, 170085, 170084, 170083, 170081, 170040, 170027, 169965, 169923, 169859, 169858, 169855, 169846, 169838, 169837, 169835

In all cases, the z-component of the primary vertex (the position of the collision within the detector) was required to be within 10 cm of the center of the ALICE detector for an event to be selected.  

\subsection{Centrality flattening}
\label{sec:CentralityFlattening}

Each event is characterized by a centrality percent.
There are 100 centrality bins ranging from 0\%  (head-on collisions) to 99\% (glancing collisions), and the centrality percent is defined such that each bin would be sampled equally using a minimum bias trigger.
The LHC11h dataset employed central and semi-central triggers in addition to the minimum bias trigger, so the centrality bins are not equally filled.
There are approximately as many events in the 0-10\% range as there are in the 10-50\% range.

While the events in the 10-50\% show an even sampling, the number of events in each of the 0-10\% centrality bins differ by a few percent.
This analysis uses a technique called centrality flattening to ensure that each of the 0-10\% bins are sampled evenly.
For each magnetic field configuration, each centrality bin is assigned a weight that brings that bin even with the bin with the smallest overall content (the 9\% bin).
We use those weights and a random number generator to determine whether or not to throw away any given event.

\subsection{Detector use}
The data in this analysis utilizes information acquired by several of ALICE's subdetectors.  
Event centrality was measured using the V0 detector with a timing cut from the Zero Degree Calorimeter. 

The neutral $\Lambda$ particle is found by looking for the pion and proton that it decays into.
The Inner Tracking System (ITS), Time Projection Chamber (TPC), Transition Radiation Detector (TRD), and Time of Flight detector (TOF) together provided global particle tracking for these daughter particles.
Information from the TPC is used to perform particle identification (PID), the act of determing if a charged track is a proton, pion, kaon, etc.
%Particle identification (PID) was performed using the TPC, as well as using the Time of Flight detector when possible.


\subsection{Analysis code}
The code used in this research to comb through the data, reconstruct $\Lambda$, and measure correlation functions was written in ROOT, an offshoot of C++ designed for statistical analysis and data visualization.
The analysis classes used for this study are checked into the ALICE Collaboration's official code bank, AliPhysics.
The code is stored in the AliPhysics/PWGCF/FEMTOSCOPY/V0LamAnalysis directory, and the classes all have the "AliAnalysisV0Lam" prefix.
The code can be run on the ALICE LEGO (Lightweight Environment for Grid Operators) Train via the macro AliPhysics/PWGCF/FEMTOSCOPY/macros/AddTaskLamLam.C.
The analysis train can run analyses for one or more people simultaneously, thus facilitating efficient usage of resources on the ALICE supercomputing grid.

For this study, the most recent analysis of an ALICE dataset was performed on 06 May, 2016 using AliRoot vAN-20160506-1.
The analysis was run using the Correlations and Fluctuations Physics Working Group (PWGCF) train with dataset LHC11h\_AOD145\_fieldconfigs.

The code used for fitting the measured correlation functions (see Section \ref{sec:FemtoFitter}) is not part of the ALICE code repository.
However, it can be found online at \url{http://github.com/jsalzwedel/FemtoFitting}.
