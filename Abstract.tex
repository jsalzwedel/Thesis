In this thesis, we present lambda baryon femtoscopy measured by the ALICE collaboration in $\sqrt{s_{NN}}=2.76$ TeV Pb-Pb collision.
Correlation functions are shown in three centrality ranges for $\Lambda\bar{\Lambda}$ and two for the combined $\Lambda\Lambda\oplus\bar{\Lambda}\bar{\Lambda}$ results.
Femtoscopy is capable of measuring the femtometer-scale spatial and temporal aspects of particle collisions by looking at relative momentum correlations among pairs of particle species.
In addition, femtoscopy can extract information about the final state interactions of the strong nuclear forces that occur between those particles.
The femtoscopic radii and scattering parameters (scattering length and effective range of interaction) of $\Lambda\Lambda\oplus\bar{\Lambda}\bar{\Lambda}$ and $\Lambda\bar{\Lambda}$ are presented.

Following recent measurements of $\mathrm{p}\bar{\Lambda}$ and $\bar{\mathrm{p}}\bar{\mathrm{p}}$ , we have expanded the standard femtoscopic fitting process to account for the correlated effects of particles that decay into $\Lambda$. % Expand this
The rest contain one or two secondary $\Lambda$.
We have estimated the $\lambda$ parameters that describe the relative contribution of each residual correlation.
Residual correlation effects are included in the fitting process by estimating their correlation strength and then smearing the relative momentum of their contribution according to the kinematics of their decay.

