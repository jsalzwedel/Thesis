Two-$\Lambda$ momentum correlation functions measured in Pb-Pb collisions at $\sqrt{s_{NN}}=2.76$ TeV by the ALICE collaboration are presented.  Results are shown for $\Lambda\Lambda$, $\bar{\Lambda}\bar{\Lambda}$ and $\Lambda\bar{\Lambda}$ pairs.


In this thesis, we will present lambda baryon femtoscopy measured by the ALICE collaboration in $\sqrt{s_{NN}}=2.76$ TeV Pb-Pb collision.
Correlation functions are shown in three centrality ranges for $\Lambda\bar{\Lambda}$ and two for the combined $\Lambda\Lambda\oplus\bar{\Lambda}\bar{\Lambda}$ results.


The standard femtoscopic fitting process has been expanded to account for the correlated effects of particles that decay into $\Lambda$. % Expand this
These are effects may be significant; it is estimated that only a quarter of pairs contain two primary $\Lambda$.
The rest contain one or two secondary $\Lambda$.
We have estimated the $\lambda$ parameters that describe the relative contribution of each residual correlation, and they are included in a table.
This technique has been used in the past to perform measurements of $\mathrm{p}\bar{\Lambda}$ and $\bar{\mathrm{p}}\bar{\mathrm{p}}$ scattering parameters.



Fit results have been found for femtoscopic radii of the pairs, with the 

, as well as for the complex scattering lengths $f_0$ and the effective range of interaction, the s-wave scattering parameters.

This approach may have applicability for measuring the poorly known interactions of other pairs, such as $\Lambda\Xi^-$, $\Lambda K^-$, or $Xi^-K^-$.