%\documentclass[10pt,a4paper]{article}
%\usepackage[latin1]{inputenc}
%\usepackage{amsmath}
%\usepackage{amsfonts}
%\usepackage{amssymb}
%\author{}
%\title{}
%\begin{document}
\tableofcontents
\listoffigures
\pagebreak
\section{Introduction}

The study of two-particle correlations at low relative momentum is commonly known as femtoscopy.  Femtoscopic analyses are capable of measuring spatio-temporal characteristics of heavy-ion collisions, with a particular emphasis made on estimating the homogeneity lengths (also called radii) of the particle-emitting source.  These analyses traditionally study pions \cite{Goldhaber:1960sf,Aamodt:2011mr} because of their ready availability and sensitivity to the spatial scale of the source.  However, measurements of heavier particles such as kaons \cite{Abelev:2012ms} and baryons \cite{Gos:2007cj} can serve to complement the pion results.  One motivation for studying an assortment of heavier particles is to test the hydrodynamic prediction that radial flow should cause the source radii to scale with the transverse mass $m_{\mathrm{T}}$ of the particles \cite{Csorgo:1995bi,Lisa:2005dd}.

Baryon--(anti)baryon correlation functions are also useful in the study of final state interactions (FSI).  Measurements of scattering lengths of pairs such as p$\Lambda$ and $\Lambda\Lambda$ are of interest for rescattering calculations, and, in the latter case, for understanding the properties of neutron stars \cite{SchaffnerBielich:2008kb,Wang:2010gr}.  Baryon-antibaryon correlations allow for the study of annihilation processes.  It has been argued that annihilation in the hadronic rescattering phase should be taken into account when determining particle yields \cite{Werner:2012xh,Karpenko:2012yf,Steinheimer:2012rd}.  This annihilation should result in an anticorrelation in baryon-antibaryon correlation functions.  Results from these femtoscopic analyses may be able to provide insight into the $p/\pi^+$ ratio measured in experiments at the LHC, which falls short of thermal model expectations \cite{Preghenella:2012eu}. 

In this note, we present two-$\Lambda$ momentum correlation functions measured in PbPb collisions at $\sqrt{s_{NN}}=2.76$ TeV by the ALICE Collaboration.  Results are shown for $\Lambda\Lambda$, $\bar{\Lambda}\bar{\Lambda}$ and $\Lambda\bar{\Lambda}$ pairs. The $\Lambda\bar{\Lambda}$ results are presented as a candidate for preliminary status. The motivation for $\Lambda$ baryon femtoscopy is to complement previous studies, as well as to probe the little-understood strong interactions of these baryons.


The primary effect seen in $\Lambda\bar{\Lambda}$ correlations should be strong final state interactions (FSI).  A suppression of the correlation function at low-$q_{\rm inv}$ is expected due to interactions in the baryon-antibaryon annihilation channel.  Suppressions of this sort have also been seen in $p \bar{p}$ interactions \cite{Gos:2007cj}, though the effects of this channel in charged particle studies can be somewhat obfuscated by Coulomb enhancement effects in the same $q_{\rm inv}$ region.  Also discussed in this note are the main influences on $\Lambda\Lambda$ correlations --- strong FSI and Fermi-Dirac statistics.  A long-term goal of this study is to measure characteristics of the interaction potential of the $\Lambda$ baryons.  More specifically, the scattering length $f_0$ and effective radius $d_0$ of the interaction can be extracted from two-$Lambda$ correlations. It should also be noted that the H-dibaryon is thought to have two $\Lambda$s as one of its decay channels \cite{PhysRevLett.38.195}.  The study of $\Lambda$ momentum correlations at ALICE may some day provide measurements of this phenomenon, though endeavors in this direction are too preliminary to report upon at this time.

\section{Data analysis}
\subsection{Data selection}
The analysis was performed on 31 July, 2013 using AliRoot v5-04-64-AN. The data were from PbPb collisions at $\sqrt{s_{NN}}=2.76$  TeV taken from the LHC11h (AOD115) pass 2 reconstruction. Runs were selected if they were listed with a pass 2 global quality of 1 ("Good run") in the Run Condition Table. Approximately 66 million combined central, semi-central and minimum bias events were analyzed. The following runs were used, which comprise both the positive (++) and negative magnetic (--) field orientations:

170593, 170572, 170388, 170387, 170315, 170313, 170312, 170311, 170309, 170308, 170306, 170270, 170269, 170268, 170230, 170228, 170207, 170204, 170203, 170193, 170163, 170159, 170155, 170091, 170089, 170088, 170085, 170084, 170083, 170081, 170040, 170027, 169965, 169923, 169859, 169858, 169855, 169846, 169838, 169837, 169835, 169591, 169590, 169588, 169587, 169586, 169557, 169555, 169554, 169553, 169550, 169515, 169512, 169506, 169504, 169498, 169475, 169420, 169419, 169418, 169417, 169415, 169411, 169238, 169167, 169160, 169156, 169148, 169145, 169144, 169138, 169099, 169094, 169091, 169045, 169044, 169040, 169035, 168992, 168988, 168826, 168777, 168514, 168512, 168511, 168467, 168464, 168460, 168458, 168362, 168361, 168342, 168341, 168325, 168322, 168311, 168310, 168115, 168108, 168107, 168105, 168076, 168069, 167988, 167987, 167985, 167920, 167915

Analysis was also performed on the LHC12a17a_fix Monte Carlo HIJING run, which corresponds to the ++ field orientation events of the LHC11h anchors.  Approximately 650 thousand events were analyzed from runs

170593, 170572, 170388, 170387, 170315, 170313, 170312, 170311, 170309, 170308, 170306, 170270, 170269, 170268, 170230, 170228, 170207, 170204, 170203, 170193, 170163, 170159, 170155, 170091, 170089, 170088, 170085, 170084, 170083, 170081, 170040, 170027, 169965, 169923, 169859, 169858, 169855, 169846, 169838, 169837, 169835

In all cases, the z-position of the primary vertex was required to be within 10 cm of the center of the ALICE detector for an event to be selected.  

\subsection{Detector use}
Both the data analysis as well as the MC analysis utilized information from several different detectors.  Event centrality was measured using the V0 detector with a timing cut from the Zero Degree Calorimeter.  The Inner Tracking System (ITS), Time Projection Chamber (TPC), Transition Radiation Detector (TRD), and Time of Flight detector together provided global tracking for the daughter particles.  Particle identification (PID) was performed using the TPC, as well as using the Time of Flight detector (TOF) when possible.

\subsection{$\Lambda$ reconstruction}
\label{sec:Recon}

During V0 reconstruction, a list of potential $\Lambda(\bar{\Lambda})$ candidates were extracted using the AliAODv0 Offline finder. The following cuts were applied for each V0's daughter tracks.

\begin{itemize}
\item The V0 was required to have no more than two daughters, and those daughters were required to have different charges.
\item The daughter tracks were required to have hits on at least 80 pad rows of the TPC.
\item The daughter protons (antiprotons) were required to have $p_{\rm T} > 0.5 (0.3)$ GeV/c, and the daughter pions required $p_{\rm T} \geq 0.16$ GeV/c.  All daughters were required to be in the range $|\eta| < 0.8$.
\item The TPC was used for PID purposes.  For proton and pion daughter candidates, the PID resolution task was used to obtain NSigma, which was required to be less than 3.
\item Daughter tracks were required to have a TOF PID NSigma value less than 4 if the TOF signal was available.
\item The two daughters are required to have a $DCA < 0.4$ cm to each other.
\item The daughters were each required to have a DCA to the primary vertex $> 0.1$ cm.
\end{itemize}

Each V0 is then subjected to the following cuts:
\begin{itemize}
\item The V0 is required to have $|\eta| < 0.8$.
\item The cosine of the pointing angle must be greater than 0.998.
\item The DCA of the V0 to the primary vertex must be less than 1 cm.
\item For a $\Lambda$, one daughter must be identified as a proton and the other as a $\pi^-$.  For a $\bar{\Lambda}$, one daughter must be identified as an antiproton and the other as a $\pi^+$.
\item For both $\Lambda$ and $\bar{\Lambda}$, the reconstructed invariant mass was required to fall within $ 1.11 < m_{\rm inv} < 1.12136$ GeV/${\rm c^2}$.
\end{itemize}

The invariant mass spectra for $\Lambda$ can be seen in Figure \ref{fig:MassLooseCut}. An approximation of the signal quality was estimated using a ratio of real and background counts.  The background was estimated using a fourth order polynomial.  The number of real $\Lambda$ was then estimated by counting the bin content and subtracting the background. The signal quality was found to be $real/(real + background) \approx 0.82$.

Figure \ref{fig:MassTightCut} shows the invariant mass spectra when tighter cuts are enforced.  The cuts increase the minimum DCA of the daughters to the primary vertex to 1 cm, reduced the max DCA of the V0 to the primary vertex to 0.8 cm, and set a minimum decay length of 2 cm.  These cuts ensure a better purity by eliminating much of the contamination of the reconstruction by primary particles.  However, there is also a significant drop ({\raise.17ex\hbox{$\scriptstyle\mathtt{\sim}$}} threefold) in the number of reconstructed $\Lambda$, as well as the number of $\Lambda$ pairs.  While these cuts afford a higher purity, the extracted correlation function was found to be less significant due to the decreased statistics.  Therefore, the looser cuts defined in Section \ref{sec:Recon} are preferred.  Further details are discussed in Section \ref{sec:Systematics}.

\subsection{Shared daughter cut}

Occasionally two or more V0s are reconstructed that both claim the same daughter track.  However, a given track can only be the daughter of, at most, one V0.  Therefore, if several reconstructed V0s all lay claim to the same daughter, at most only one of them can be a true V0.  It was decided to employ a cut that would compare characteristics of the different V0s to determine which of them was most likely to be the true parent particle.  After the best V0 was found, the competing V0s were removed from the list of V0 particles and not used during same- or mixed-event pair construction.

A study was done using MC event truths to determine which cut would be most successful at removing fraudulent V0s.  The following characteristics were examined:

\item Closeness of invariant mass to the PDG mass value.
\item DCA of the daughter particles to each other.
\item Cosine of the pointing angle.
\item DCA of the V0 to the primary vertex.

Based on the MC truth study, the ... cut was found to successfully keep true V0s and remove fake V0s ...\% of the time.  

On average there were two competing particles per every ... events in the MC study.  When applied to analysis of the LHC11h data, the rate was two per ....

In order to keep spurious correlated signals from entering the correlation functions, a cut on daughter particles was implemented.  After all the V0s in an event were reconstructed, the V0s were checked to see if they shared any daughters.  Those V0s that shared a daughter were compared via several characteristics:

As this shared-daughter cut process was done sequentially through the ... an iterative process was developed to ensure... 

(example: V0s "A" and "B" both share a daughter, and V0s "A" and "C" share a different daughter.  "A" has a better DCA than "B", so "B" is removed.  Then "C" is found to have a better DCA than "A", so "A" is removed.  Finally, "B" is returned because it no longer competes for a daughter).

This cut helps remove a splitting-like effect (see sec \ref{sec:CFconstruct) for more on splitting), wherein one true V0 has each of its daughters claimed by two separate fake V0s.  By the nature of the reconstruction cuts, all three of the V0s will be close in momentum space.  Therefore a pair constructed from the two fake V0s will contaminate the low relative-momentum region of the correlation function.  


\subsection{Correlation function construction and pair cuts}
\label{sec:CFconstruct}

This analysis studies two-particle correlations as a function of the one dimensional relative momentum $q_{\mathrm{inv}}=2k^*=\abs{P(m^2_1-m^2_2)/P^2 - q}$, where $P$ and $q$ are the four-vector momentum sum and difference, respectively, and $m_1$ and $m_2$ are the masses of the two particles.  The correlation function is constructed as $C(q_{\rm inv}) = A(q_{\rm inv})/B(q_{\rm inv})$, where $A(q_{\rm inv})$ is the two-particle distribution of the given event and $B(q_{\rm inv})$ is the distribution of a reference event.  In practice, the signal event histogram $A(q_{\rm inv})$ is constructed by binning the $q_{\rm inv}$ value of each pair of particles in a single event, and repeating this for each event.  The reference (or background) histogram is constructed by binning $q_{\rm inv}$ for pairs of particles taken from different events.  Care is taken to ensure that pairs are mixed from events with similar centralities (bin width of 5\%) and primary vertex z-position (2 cm bin width).  There are five mixed events for each real event.

Femtoscopic studies look at the relative momentum of particles, and often the most interesting physics lies at very low relative momentum (around 0.2 GeV/c and lower).  As a result, two-track reconstruction effects such as track splitting and track merging, both of which occur for tracks with similar momenta and trajectories, can have a large effect on the final results.  Track splitting means that the track left by a single charged particle is reconstructed as two separate tracks. Track merging is when the tracks of two separate particles are reconstructed as a single track.  

In this analysis, two-track reconstruction effects are combatted via a cut on the average separation of daughter tracks from different lambdas.  The average separation distance of two daughter tracks are computed at nine different radii of the TPC.  The tracks are propagated iteratively using their local coordinates and the PropagateTo function, and their global positions are determined at 85, 105, 125, 145, 165, 185, 205, 225, and 245 cm.  Initial iteration is performed in 1 cm steps, though a secondary propagation is done in 0.1 cm steps for better precision.  Same- and mixed-event pairs are binned according to this separation distance.  To obtain an estimate of the merging/splitting effects both distributions are then scaled by the number of pairs at high average separation distance (10+ cm), and a correlation function (same event pairs/ mixed event pairs) is created.  Looking at Figure \ref{fig:Merging}, one can see that splitting effects are relevant out to about 1 cm, as evidenced by a relative abundance of same-event pairs.  Merging is visible out to about 3 cm, as evidenced by a dearth of same-event pairs in that range.  To address these effects, both same- and mixed-event V0 pairs are cut if they have a like-sign daughter pair with average separation less than 3 cm.

To ensure that average separation distributions of mixed-event pairs are comparable to the same-event distributions, it is necessary to perform a shifting of the primary vertex for mixed events.  Doing so allows the track separations to be calculated as though both events had the same primary vertex.  Without this correction, the mixed-event distributions are biased by differences in the primary vertex location.  One could imagine an extreme example of this if one used a 20 cm wide z-vertex bin.  In that case, tracks from different events could be shifted relative to each other by as much as 20 cm.  For this analysis with 2 cm z-vertex binning, and the uncorrected average separation correlation is shown in Figure \ref{fig:Merging}.  In contrast, Figure \ref{fig:CorrectedMerging} shows the primary-vertex corrected correlation.  To get a clear account of the difference between the two, we can look at Figure \ref{fig:AvgSepRatio}, which shows the ratio of the uncorrected to corrected correlations.  The enhancement in the... shows that...

Finally, when constructing pairs of V0s from the same event, care is taken to ensure that the two V0s do not share a daughter with the same track id. 

\section{Results and Systematic Uncertainties}
\label{sec:ResultsSystematicsData}

..... Discussion of correlation function data results and systematic uncertainties.  ......
%\section{Results}
%\label{sec:Results}

Figure \ref{fig:CFMixCentralities} shows $\Lambda\bar{\Lambda}$ correlation functions versus $q_{\rm inv}$ for three different centralities ranges.  The correlations are normalized to unity in the $ 0.3 < q_{\rm_{inv}} < 0.5$ GeV/c range.  Each correlation shows clear suppression in the low-$q_{\rm_{inv}}$ region, which is considered an effect of the baryon-antibaryon annihilation channel.  The strength of the correlation effect increases with more peripheral events, an indication that the emitting source size is shrinking for those events.

Figure \ref{fig:CF} shows correlation functions constructed in the 0-10\% centrality range for $\Lambda\Lambda$ and $\bar{\Lambda}\bar{\Lambda}$ pairs.  Both correlation functions exhibit an enhancement in the low-$q_{\rm inv}$ range of 0.04 - 0.2 GeV/c.  This may be due to attractive final state interactions.  The effects of FSI and quantum statistics are expected to be seen in roughly the same $q_{\rm inv}$ range, and it is unclear at this time to what extent their effects should be competing. The enhancement of the $\bar{\Lambda}\bar{\Lambda}$ correlation function does fall off in the lowest bins, though the statistical uncertainties are such that no significant statement about the effects of quantum statistics can be made at this time.

Efforts to properly obtain fit parameters are still under way.  

%\section{Systematic uncertainties}
%\label{sec:Systematics}

This analysis is in a preliminary phase where fit parameters such as source radii and correlation strength are unavailable.  Therefore a discussion of systematic effects is at this time limited to showing how different sets of cuts change the qualitative aspects of the correlation function plot.  The effects of two such cuts are discussed here.  An attempt is made to quantify the systematic uncertainty between these different cuts for the $\Lambda\bar{\Lambda}$ correlation function.

Two-track reconstruction effects were discussed in Section \ref{sec:CFconstruct}. Figure \ref{fig:CFNoMergeMix} shows the three centralities of $\Lambda\bar{\Lambda}$ correlation functions measured without the average separation cut in place.  Figure \ref{fig:CFNoMerge} shows the equivalent plot for the $\Lambda\Lambda$ and $\bar{\Lambda}\bar{\Lambda}$ correlations. Subtle differences are visible in the $q_{\rm inv} < 0.2$ GeV/c range, with the most drastic changes lying in the lowest bins where the statistics are the weakest.  Except for the lowest bin, the changes all fall within error bars of the points in Figures \ref{fig:CFMixCentralities} and \ref{fig:CF}.  Nonetheless, it is clear that including the two-track cuts  shifts the low-$q_{\rm inv}$ bins down by a noticeable amount, a sign that splitting effects are being removed.  This consistent downward shift indicates that two-track effects are a significant source of systematic error in this study.

As mentioned in Section \ref{sec:Recon}, two sets of reconstruction cuts (loose and tight) have been studied in detail.  Figure \ref{fig:CFMixCentralities} was created using loose reconstruction cuts.  For comparison, Figure \ref{fig:CFTightCutsMix} shows the same correlations with tighter reconstruction cuts employed.  Figure \ref{fig:CFTightCuts} shows a similar plot for the $\Lambda\Lambda$ and $\bar{\Lambda}\bar{\Lambda}$ correlations.  It is apparent that the statistics are much worse under the tigher cuts, though qualitative similarities to the other plots remain, e.g.\ the centrality dependence of the $\Lambda\bar{\Lambda}$ annihilation and the enhancement of the $\Lambda\Lambda$ and $\bar{\Lambda}\bar{\Lambda}$ at low-$q_{\rm inv}$.

A convervative estimate of the systematic errors for the $\Lambda\bar{\Lambda}$ correlation function and are shown in Figure \ref{fig:CFMixSystematics}.  The systematic errors were computed by taking the absolute value of the difference between Figure \ref{fig:CFMixCentralities} and Figure \ref{fig:CFTightCutsMix} for each $q_{\rm inv}$ bin.  The results were then fit with a 5th order polynomial.  The value of the polynomial was determined at the center of each $q_{\rm inv}$ bin.  This value was halved and then applied as a symmetric systematic error for that particular $q_{\rm inv}$ bin.  The data points and statistical errors are those from Figure \ref{fig:CFMixCentralities}.





\section{Fitting}
\subsection{Analytical Model}
The correlation functions can be calculated analytically \cite{lednicky82} using 
\begin{equation}
\label{eqLednicky}
C(k^*)= 1+ \displaystyle\sum\limits_{S}\rho_S\left[\frac{1}{2}\abs{\frac{f^S(k^*)}{R}}^2(1-\frac{d_0^S}{2\sqrt{\pi}R})+\frac{2\Re f^S(k^*)}{\sqrt{\pi}R}F_1(2k^*R)-\frac{\Im f^S(k^*)}{R}F_2(2k^*R)\right]
\end{equation}
where $F_1(z) = \int_0^z \! \mathrm{d}x \, e^{x^2-z^2}/z$,  $F_2(z) = (1-e^{-z^2})/z$, and $R$ is the source size. We assume the particles are produced unpolarized, with $\rho_{\mathrm{S}}$ being the fraction of pairs in each total spin state S.  $f^S(k^*)=(1/f_0^S+\frac{1}{2}d_0^Sk^*-ik^*)^{-1}$ is the spin-dependent scattering amplitude, though this analysis measures only spin-averaged values. The scattering amplitude is dependent upon the complex scattering length $f_0^S$.  The real part of the scattering amplitude can contribute either a positive or a negative correlation, but either way the effect is relatively narrow in $k^*$ (on the order of about one hundred MeV/$c$).  The imaginary part of the scattering amplitude accounts for inelastic processes of baryon-antibaryon annihilation.  Introducing a non-zero imaginary part to the scattering length produces a wide (hundreds of MeV/$c$) negative correlation.  For identical and charged particles, additional terms \cite{Aamodt:2011kd} are necessary to account for quantum interference and the Coulomb interaction, respectively.
  
   the effective radius of interaction $d_0^S$ comes from scattering theory \cite{...}, and it loosely represents half the length of the interaction distance.

\subsection{Fit Parameters}
\label{sec:Parameters}

For pion, kaon, and proton femtoscopic analyses, the scattering lengths and effective radius of interaction are well specified by years of scattering data, and they can be fixed to the known values.  In the case of lambda-(anti-)lambda femtoscopy, the interaction is heretofore large unstudied.  For this analysis, the scattering lengths and effective radius are left as free parameters.  Another free parameter is the $\lambda$ parameter, which is roughly a measure of the pair purity.  For example, $\lambda_{\Lambda\Lambda}$ would correspond to the ratio of number of true primary $\Lambda\Lambda$ pairs to the total number of pairs in the correlation function. The remaining fraction is comprised of all the other combinations of pair types, such as $\Lambda$-fake $\Lambda$ pairs, or pairs between primary and secondary lambdas.  The latter case will be described further in Section \ref{sec:Residual}.  The last fit parameter is $R_{\rm inv}$, which characterizes the one dimensional size of the emitting source.  



\subsection{Residual Correlations}
\label{sec:Residual}

Attempts at fitting correlation functions should try to account for residual correlations between the particle being studied and other particles that might decay into the studied particle.  Various particles can decay into $\Lambda$ baryons, such as $\Sigma^{\rm 0}$, $\Xi^{\rm 0}$, and $\Omega^{\rm -}$.  The decay momentum between these particles and their daughter $\Lambda$ is small, such that daughter and parent carry very similar momenta.  As a result, relative-momentum correlation functions between primary particles and secondary particles such as $\Lambda$ should be sensitive to the interactions between the primary particles and the parent particles.  For example, the correlation between primary $\Lambda$ and primary $\Sigma^{\rm 0}$ should be visible, though somewhat smeared out, in the correlation between primary $\Lambda$ and secondary $\Lambda$.  Furthermore, these secondary $\Lambda$ are often difficult to distinguish from primary $\Lambda$.  Therefore, measurements of two-$\Lambda$ correlation functions will contain a mixture of primary correlations and feeddown correlations, and an attempt to parameterize the correlation functions should take this into account.  

One example of this type of parameterization can be seen in the recent proton femtoscopy results from ALICE PbPb collisions \cite{Szymanski:2012AN}.  In that study, initial attempts to fit a purely primary proton-proton correlation function failed to reproduce the shape of the data.  However, when further parameterization was introduced to take into account proton-proton correlations where one proton was primary and one came from a $\Lambda$ decay, the fit was remarkably successful.  An analysis of residual correlations is not yet available for this $\Lambda$ study, though it is a work in progress.  Future femtoscopic analyses may also find this approach useful.

%%%%%%%%%%%%%%%%%%%%%%%%%
%%%%%%%%%%%%%%%%%%%%%%%%%
%%%% Replace references to protons with references to lambdas/sigmas
%%%%%%%%%%%%%%%%%%%%%%%%%
We attempt to quantify the residual contamination in our correlation functions by simultaneously fitting the data for both the primary correlation function and the residual correlation function.  For pp correlations with $\mathrm{p\Lambda}$ feeddown, we fit our data using 
\begin{equation}
\label{eq:Residual}
C_{\mathrm{meas}}(k^*_{\mathrm{pp}})= 1 + \lambda_{\mathrm{pp}}[C_{\mathrm{pp}}(k^*_{\mathrm{pp}})-1]+\lambda_{\mathrm{p\Lambda}}[C_{\mathrm{p\Lambda}}(k^*_{\mathrm{pp}})-1],
\end{equation}
where $$C_{\mathrm{p\Lambda}}(k^*_{\mathrm{pp}}) \equiv \displaystyle\sum\limits_{k^*_{\mathrm{p\Lambda}}}C_{\mathrm{p\Lambda}}(k^*_{\mathrm{p\Lambda}})T(k^*_{\mathrm{p\Lambda}},{k^*_\mathrm{pp}}),$$ $C_{\mathrm{p\Lambda}}(k^*_{\mathrm{p\Lambda}})$ is the $\mathrm{p\Lambda}$ correlation function calculated from Eq.~(\ref{eqLednicky}), and $T$ is a THERMINATOR \cite{Chojnacki:2011hb} transform matrix, which generates pairs of particles and determines kinematically how the $k^*$ of the pair transforms when one of the particles decays.
%%%%%%%%%%%%%%%%%%%%%%%%



Therminator plot

Equation for adding two correlations together.

\section{Fitting Results and Systematics}
\label{sec:FittingSystematics}
Discussion of current status of fit results and systematic uncertainties.  Report measured systematics as well as possible sources of systematics yet to be explored.

\subsection{Estimates of $\lambda$ parameters}

Lambda parameter equations.  Caveat that these equations apply on an event by event basis, and so they would get averaged out over many events and different multiplicities.  I.e. the ratio of total lambda-lambda pairs to all pairs in the whole correlation function is not constructed by taking the formula for lambda-lambda pairs and putting into total lambda yields.
Yield estimates from thermal model.
Efficiency calculation.
Lambda parameter estimates.

\section{Summary}
Results have been shown for correlation functions $\Lambda\Lambda$, $\bar{\Lambda}\bar{\Lambda}$ and $\Lambda\bar{\Lambda}$ pairs.  The results show that like-particle pairs are enhanced at low relative momentum.  The low-$q_{\rm inv}$ suppression of the $\Lambda\bar{\Lambda}$ correlations in each of the three centrality bins may indicate pair annihilation processes reminiscent of those reported in other studies.

\begin{figure}[hbtp]
\includegraphics[scale=0.6]{2012-Aug-07-LamInvMass_Perf.pdf}
\caption[Invariant mass distribution for reconstructed $\Lambda$]{Invariant mass distribution for reconstructed $\Lambda$ using the main analysis cuts.  The red line shows a fourth order polynomial fit to the background, which is used to estimate the number of real and accidental $\Lambda$.  The estimated ratio of real $\Lambda$ to all reconstructed $\Lambda$ in the signal region ($ 1.11 < m_{\rm inv} < 1.12136$ GeV/${\rm c^2}$) shown in this plot is approximately 0.82.  Candidate for performance status.}
\label{fig:MassLooseCut}
\end{figure}

\begin{figure}[hbtp]
\includegraphics[scale=0.5]{11h_V0_MassLam_TightCuts.png}
\caption[Invariant mass distribution for reconstructed $\Lambda$ using tight cuts]{Invariant mass distribution for reconstructed $\Lambda$ using tight cuts.  Here, $sig/(sig+bkg) \approx 0.92$.}
\label{fig:MassTightCut}
\end{figure}

\begin{figure}[hbtp]
\includegraphics[scale=0.5]{Qfac_MixedEvent.png}
\caption[$qfactor$ vs $q_{inv}$ distribution of mixed-event pairs]{Scaled $qfactor$ vs $q_{inv}$ distribution of mixed-event proton-daughter pairs.  Since the daughters come from two different events, two-track reconstruction effects are absent.}
\label{fig:QfacMixed}
\end{figure}

\begin{figure}[hbtp]
\includegraphics[scale=0.5]{Qfac_SameEvent.png}
\caption[$qfactor$ vs $q_{inv}$ distribution of same-event pairs]{Scaled $qfactor$ vs $q_{inv}$ distribution of same-event pairs.  Note the region of excess pairs (relative to Fig \ref{fig:QfacMixed}) at $qfactor \approx 0.45$ and $q_{inv} \leq 0.1$ GeV/c.  To excise splitting effects when constructing the correlation functions, V0 pairs are removed if their daughters have a $qfactor \geq 0.2$.}
\label{fig:QfacSame}
\end{figure}

\begin{figure}[hbtp]
\includegraphics[scale=0.6]{2012-Aug-07-TwoTrack_Perf.pdf}
\caption[Correlation function showing two-track reconstruction effects]{Correlation function versus average separation distance in the TPC for proton daughters.  Constructed using same-event pairs over mixed-event pairs.  Both splitting (enhancement) and merging (suppression) effects are visible.  Candidate for performance status.}
\label{fig:Merging}
\end{figure}

\begin{figure}[hbtp]
\includegraphics[scale=0.7]{LamALamCF_LooseCuts.pdf}
\caption[Correlation functions versus $q_{\rm inv}$ for $\Lambda\bar{\Lambda}$ pairs in three centrality ranges.]{Correlation functions versus $q_{\rm inv}$ for $\Lambda\bar{\Lambda}$ pairs.  Results are shown for the 0-10\%, 10-40\%, and 40-60\% centrality range.}
\label{fig:CFMixCentralities}
\end{figure}

\begin{figure}[hbtp]
\includegraphics[scale=0.7]{CFs_main_note.pdf}
\caption[Correlation functions versus $q_{\rm inv}$ for $\Lambda\Lambda$ and $\bar{\Lambda}\bar{\Lambda}$ pairs.]{Correlation functions versus $q_{\rm inv}$ for $\Lambda\Lambda$ and $\bar{\Lambda}\bar{\Lambda}$ pairs.  Results are shown for the 0-10\% centrality range.}
\label{fig:CF}
\end{figure}

\begin{figure}[hbtp]
\includegraphics[scale=0.7]{MixCFs_No_MergeCut.pdf}
\caption[Correlation functions versus $q_{\rm inv}$ for $\Lambda\bar{\Lambda}$ pairs in three centrality ranges. No merging cuts]{Correlation functions versus $q_{\rm inv}$ for $\Lambda\bar{\Lambda}$ pairs (without cut on the average separation distance of the daughter pairs).  Results are shown for the 0-10\%, 10-40\%, and 40-60\% centrality range.}
\label{fig:CFNoMergeMix}
\end{figure}

\begin{figure}[hbtp]
\includegraphics[scale=0.7]{Mix_Tight_cuts_ANote.pdf}
\caption{Correlation functions versus $q_{\rm inv}$ for $\Lambda\bar{\Lambda}$ pairs in three centrality ranges. Tight cuts.}
\label{fig:CFTightCutsMix}
\end{figure}

\begin{figure}[hbtp]
\includegraphics[scale=0.7]{CFs_noMergeCut_note.pdf}
\caption[Correlation functions versus $q_{\rm inv}$ for $\Lambda\Lambda$ and $\bar{\Lambda}\bar{\Lambda}$ pairs. No merging cuts]{Correlation functions versus $q_{\rm inv}$ for $\Lambda\Lambda$ and $\bar{\Lambda}\bar{\Lambda}$ pairs (without cut on the average separation distance of the daughter pairs).  Results are shown for the 0-10\% centrality range.}
\label{fig:CFNoMerge}
\end{figure}

\begin{figure}[hbtp]
\includegraphics[scale=0.7]{CFs_tightcut_note.pdf}
\caption[Correlation functions versus $q_{\rm inv}$ for $\Lambda\Lambda$ and $\bar{\Lambda}\bar{\Lambda}$ pairs.  Tight cuts]{Correlation functions versus $q_{\rm inv}$ for $\Lambda\Lambda$ and $\bar{\Lambda}\bar{\Lambda}$ pairs using tight reconstruction cuts.  Results are shown for the 0-10\% centrality range.}
\label{fig:CFTightCuts}
\end{figure}

\begin{figure}[hbtp]
\includegraphics[scale=0.7]{2012-Jul-25-LamALamPrelim.pdf}
\caption[Correlation functions versus $q_{\rm inv}$ for $\Lambda\bar{\Lambda}$ pairs in three centrality ranges with systematic errors.]{Correlation functions versus $q_{\rm inv}$ with systematic errors for $\Lambda\bar{\Lambda}$ pairs. Systematics are computed by fitting the absolute value of the difference between Figure \ref{fig:CFMixCentralities} and Figure \ref{fig:CFTightCutsMix} with a 5th order polynormial. Results are shown for the 0-10\%, 10-40\%, and 40-60\% centrality range.}
\label{fig:CFMixSystematics}
\end{figure}

\bibliographystyle{plain}
\bibliography{bibfile}





%\end{document}